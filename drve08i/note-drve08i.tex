\documentclass{article}
\def\ntitle {Notes for \texttt{drve08i}}
% \def\nauthor{ } % default author is Zhengbo Zhou
% \def\notes{ } % default is the submitted version
\def\ndate{ 21 January 2023 } % default is today's date
% def\needcrop{ } % crop for easy previewing
\input{/Users/clement/bibtex/clem.tex}

\renewcommand{\comment}[1]{\textsf{\color{blue} #1}}

\begin{document}
\maketitle
% \tableofcontents
\thispagestyle{firstpage} 
%%%%%%%%%%%%%%%%%%%%%%%%%%%% MAIN ARTICLE %%%%%%%%%%%%%%%%%%%%%%%%%%%%

\begin{abstract}
  \emph{The paper mentioned in this notes is \cite{drve08i}.}
  This paper gives a detailed preconditioning method on the classical
  cyclic one-sided Jacobi SVD algorithm. 
  % 1. What did the authors try to accomplish?
  The authors achieved a fast (comparable with the QR algorithm) and
  accurate Jacobi SVD algorithm for any matrix $X$. Moreover, the authors
  provided lots of intuitions and motivations on the QR preconditioning
  methods not only in the sense of reducing dimension, but also in the
  sense of faster convergence. 

  % 2. What were the key elements of the approach 
  The key element of this approach is the efficient QR preconditioners. The
  authors not only focused on how to reduce the dimension (therefore
  flops), but also on detecting the ``digonality'' of the implicit
  diagonalization, special pivoting to boost the convergence, etc.
  Moreover, the exciting part is how to construct the right handed singular
  vector. Normally, one can intuitively accumulate the Jacobi rotations to
  get the right handed singular vector matrix. However, the authors
  considered the construction of that singular vector matrix as a solution
  of the matrix equation $AX = B$ where $X$ is unknown, $B$ is any matrix
  and $A$ is triangular. This can be done using \texttt{STRSM} which is
  more efficient than just simply multiplying the orthogonal matrices
  together. 
\end{abstract}

\section{Introduction}
Jacobi first notice this method while solving the following system of equations
\begin{equation}\notag
  \left\{
    \begin{array}{cccccccc}
      \{(a, a)-x\} \alpha & + & (a, b) \beta & +
      & (a, c) \gamma & +\cdots+ & (a, p) \tilde{\omega} & =0 \\
      (b, a) \alpha & + & \{(b, b)-x\} \beta & +
      & (b, c) \gamma & +\cdots+ & (b, p) \tilde{\omega} & =0 \\
      (c, a) \alpha & + & (c, b) \beta & +
      & \{(c, c)-x\} \gamma & +\cdots+ & (c, p) \tilde{\omega} & =0 \\
      (p, a) \alpha & + & (p, b) \beta & +
      & (p, c) \gamma & +\cdots+ & \{(p, p)-x\} \tilde{\omega} & =0
    \end{array}
  \right.
\end{equation}
where the coefficients are symmetric, i.e. $(a,b) = (b,a), (a,c) = (c,a),
\dots, (b,c) = (c,b)$ etc. In our notation 
\begin{equation}\notag
  H =
  \begin{pmatrix}
    (a,a) & (a,b) & (a,c) & \dots & (a,p) \\
    (b,a) & (b,b) & (b,c) & \dots & (b,p) \\
    (c,a) & (c,b) & (c,c) & \dots & (c,p) \\
    \vdots {} & \vdots & \vdots & \ddots & \vdots \\
    (p,a) & (p,b) & (p,c) & \cdots & (p,p) 
  \end{pmatrix}  =  (H_{ij})^n_{i,j = 1} ,  
  \quad u =
  \begin{pmatrix}
    \alpha \\ \beta \\ \gamma \\ \vdots \\ \tilde \omega 
  \end{pmatrix}, \quad
  (H = H^T)
\end{equation}
we obtain the equation $(H - \lambda I)u = 0$, which is to be solved for
$\lambda$ and $u\neq 0$. The algorithm starts with $H^{(0)} = H$ and then
it generates a sequence of congruences, $H^{(k+1)} =
(V^{(k)})^TH^{(k)}V^{(k)}$, where $V^{(k)}$ is the plane rotation,
i.e. $V^{(k)}$ differs from the identity only at the position $(p_k,p_k),
(p_k,q_k),(q_k,p_k),(q_k,q_k)$ where
\begin{equation}\notag
  \begin{pmatrix}
    V_{p_k,p_k}^{(k)} & V_{p_k,q_k}^{(k)} \\
    V_{q_k,p_k}^{(k)} & V_{q_k,q_k}^{(k)} 
  \end{pmatrix} =
  \begin{pmatrix}
    \cos(\phi_k) & \sin(\phi_k) \\ -\sin(\phi_k) & \cos(\phi_k) 
  \end{pmatrix}. 
\end{equation}
The angle $\phi_k$ is determined to annihilate the $(p_k,q_k)$ and
$(q_k,p_k)$ positions in $H^{(k)}$, and it is carefully chosen such that we
always take $\phi_k$ as the smaller roots which ensures the convergence.

\subsection{History}
The advent of electronic computing machinery opened a new chapter in
numerical mathematics and the Jacobi method becomes attractive tool for
computing the eigenvalues of symmetric matrices. Around 1950, Goldstine,
Murray and von Neumann~\cite{gmv59} rediscover the method and had detailed
implementation and error analysis. It was used by \bf{Hessenberg} in 1940,
by \bf{Wilkinson} in 1947 at the NPL, and by \bf{Bodewig} in 1949.

\subsection{A Bit Word on the Jacobi}
The convergence is easily monitored by using the off-norm,
\begin{equation}\notag
  \Omega(H) = \sqrt{\sum_{i\neq j} H^2_{ij}},
\end{equation}
for which one easily shows the monotonicity 
\begin{equation}\notag
  \Omega(H^{(k+1)}) = \Omega(H^{(k)})-2(H^{(k)})^{2}_{p_k,q_k}\leq \Omega(H^k)
\end{equation}
Under suitable pivoting strategy, the sequence $(H^{(k)})_{k = 0}^{\infty}$
converges to $\Lambda$ and the accumulated product
$V\iter{0}V\iter{1}\cdots V\iter{k}\cdots$ converges to the orthogonal
matrix $V$ of eigenvectors of $H$, $HV = V\Lambda$. 

\subsection{Convergence} 

Jacobi proved that the convergence of a greedy approach that annihilates
the absolutely largest off-diagonal entry at each step. The greed strategy
is usually replaced with the row-cyclic strategy, first used
by~\cite{greg53}, which is periodic and in one full sweep of $n(n-1)/2$
rotations it rotates row-by-row at the pivot positions
$(1,2),(1,3),\dots,(1,n); (2,3),\dots,(2,n); \dots; (n-1,n)$. Similarly,
column-cyclic strategy scans the strict upper triangular of the matrix in
column-by-column fashion. Forsythe and Henrici \cite{fohe60} proved the
convergence of the serial Jacobi algorithm. \cite{scho61} and \cite{wilk62}
proved the quadratic convergence in case of the simple eigenvalues and
Hari~\cite{hari91} extended the result to the general case of multiple
eigenvalues. Mascarenhas~\cite{masc95} and Rhee Hari~\cite{rhha93} showed
that certain modification of row-cyclic strategy achieves cubic
convergence. Rutishauser~\cite{ruti66} described detailed implementation of
the method for real symmetric matrices.

Goal: \bf{Mathematical software implementing the new algorithm should be
  numerically sound and competitive in efficiency with the LAPACK's
  implementations of the QR and the divide-and-conquer algorithms, or any
  other bidiagonalization based procedure.}

Previous Works
\begin{itemize}
 \item Jacobi method is more accurate than QR. \cite{deve92} 
 \item Computing the Singular and the Generalized Singular Values.
   \cite{drma94_PHD} 
\item Implementation of Jacobi rotations for accurate singular value
  computation in floating point arithmetic. \cite{drma97}
\item A posteriori computation of the singular vectors in a preconditioned
  Jacobi SVD algorithm \cite{drma99} 
\end{itemize}

The development is divided into the following 

(i) Substantial modifications of the classical Jacobi SVD algorithm are
necessary to reduce its complexity. Improve the convergence by
preconditioner. Seek for zero and almost diagonal situation.

(ii) The design of the algorithm should be open for further improvements.
It should be based on building blocks which can benefit from the
development of basic matrix computational routines (BLAS, LAPACK etc.) and
blocked versions of Jacobi rotations, but without trading the numerical
accuracy. 

The structure of the paper:
\begin{itemize}
\item Section 2: Quick introduction to the numerical analysis of the
  symmetric definite eigenvalue problem and the SVD.
\item Section 3: Description of the preconditioning.
\item Section 4: Discuss whether put $A^T$ or $A$ into the algorithm. 
\item Section 5: Basic structure of the new Jacobi SVD problem.
\end{itemize}

\section{Accuracy and Backward Stability in SVD Computation}

Let $H$ be $n\times n$ symmetric positive definite matrix. An algorithm
that computes the singular values $\wt \sigma_1\geq \cdots \geq \wt
\sigma_n$ of a nearby matrix $A + \delta A$, where $\norm{\delta
  A}\leq\epsilon A$ with some small $\epsilon$. Weyl's theorem implies 
\begin{equation}\notag
  \max_{i=1:n}\frac{|\wt \sigma_i-\sigma_i|}{\norm{A}}\leq
\frac{\norm{\delta A}}{\norm{A}} \le \epsilon,
\end{equation}
and in the case of full column rank $A$, $A+\delta A = (I+\delta A
A^{\dagger})A$ yields the bound 
\begin{equation}\label{eq:fwd-error-sv}
  \max_{i=1:n}\frac{|\wt \sigma_i-\sigma_i|}{\sigma_i} \leq 
\norm{\delta A A^{\dagger}}.
\end{equation}

In general, $\delta A$ is dense with no particular structure which means
that in the expression $\delta AA^{\dagger} = \delta
AV\Sigma^{\dagger}U^T$, large singular values of $A^{\dagger} =
V\Sigma^{\dagger}U^T$ may get excited by $\delta A$.

\begin{example}
  Bidiagonalization based SVD algorithm produces $\delta A$ for which the
  best general upper bound is $\norm{\delta A} \leq \epsilon_B\norm{A}$
and thus
\begin{equation}\notag
  \norm{\delta AA^{\dagger}}\le \frac{\norm{\delta A}}{\norm{A}}\kappa(A)
\leq \epsilon_B\kappa(A),\quad \kappa(A) = \norm{A}\norm{A^{\dagger}}.
\end{equation}
In other words, if $\sigma_1\geq \dots \geq \sigma_k\gg \sigma_{k+1} \ge
\dots \ge \sigma_n > 0$, then the dominant singular values
$\sigma_1,\dots,\sigma_k$ will be computed accurately in the sense that 
\begin{equation}\notag
  \max_{i=1:k}\frac{\abs{\wt \sigma_i-\sigma_i}}{\sigma_i}\leq
\frac{\norm{\delta A}}{\norm{A}} \frac{\sigma_1}{\sigma_k}, \qquad 
\norm{\delta A}\le O(\epsilon)\norm{A},
\end{equation}
but the smallest one will have the error bound 
\begin{equation}\notag
  \frac{\abs{\wt \sigma_n - \sigma_n}}{\sigma_n} \le 
\frac{\norm{\delta A}}{\norm{A}} \frac{\sigma_1}{\sigma_n}.
\end{equation}
Thus, if for some $i$, it holds $\sigma_i \approx \varepsilon \sigma_1 \ll
\sigma_1 = \norm{A}$, then we \emph{cannot expect any correct digit in $\wt
  \sigma_i$}
\end{example}

\bf{Why should we care about small singular values?} The question can also
become, how small is small? What if we there is no clear cut such that we
can make a low rank approximation, or we meet the devil's stairs,
\begin{equation}\notag
  \sigma_1 \ge \dots \ge \sigma_{k_{1}} \gg 
\sigma_{k_1+1} \ge \dots \ge \sigma_{k_2} \gg
\sigma_{k_2+1} \ge \dots \ge \sigma_{k_3} \gg
\sigma_{k_3+1} \ge \dots \gg \sigma_{n}
\end{equation}
with $\sigma_{k_j+1}/\sigma_{k_j} \approx O(\epsilon)$. 

\begin{quote}
  In many important applications the smallest singular values are really
  the noise excited by the uncertainty in the data and computing them to
  high relative accuracy is meaningless and illusory. In such cases, the
  Jacobi SVD algorithm has no advantage with respect to accuracy.

  But given the adaptivity and  of the Jacobi algorithm to modern serial
  and parallel computing machinery, it is exciting and challenging task to
  improve the efficiency of the algorithm to make it competitive with
  bidiagonalization based algorithms in terms of speed and memory usage,
  even if the high relative accuracy of the smallest singular values is not
  an issue. 
\end{quote}

\subsection{Basic floating point error analysis}


\textbf{Fact 1.} If numerically orthogonal matrix $\wt Q$ ($\norm{\wt Q\tp \wt
  Q - I} \ll 1$) is applied to an $m\times 1$ vector in floating point
arithmetic. Then $computed(\wt Q x) = \wh Q(x+\delta x)$, where $\wh Q$ is
orthogonal matrix close to $\wt Q$, and $\norm{\delta x} \leq \epsilon
\norm{x}, \epsilon \leq f(k)\varepsilon $. Here $k$ is the number of
coordinate directions changed under the action of $\wt Q$.

\noindent\textbf{Fact 2.} If numerically orthogonal transformations $\wt
Q_1,\dots, \wt Q_p$ are applied to \emph{disjoint parts of an vector $x$}:
\begin{equation}\notag
  x\mapsto y  = (\wt Q_1\oplus \cdots\oplus \wt Q_p)x, \quad x = 
  \begin{pmatrix}
    x^{(1)} \\ \vdots \\ x^{(p)}
  \end{pmatrix}, 
computed(\wt Q_ix^{(i)}) = \wh Q_i(x^{(i)} + \delta x^{(i)}),
\end{equation}
then we have 
\begin{equation}\notag
  \wt y \equiv computed(y) = (\wh Q_1\oplus \cdots \oplus \wh Q_p)(x
  + \delta x), \qquad 
\frac{\norm{\delta x}}{\norm{x}} \leq \max_{i = 1:p}\frac{\norm{\delta
    x^{(i)}}}{\norm{x^{(i)}}}. 
\end{equation}

\noindent\textbf{Fact 3.} If $\wt Q_1,\dots,\wt Q_r$ are numerically
orthogonal transformations, and if we need to compute $y = \wt Q_r\dots \wt
Q_1 x$, then the computed approximation $\wt y$ satisfies 
\begin{equation}\notag
  \wt y = \wh Q_r\dots \wh Q_1(x + \delta x), \quad 
  \norm{\delta x} \leq ((1+\epsilon)^r - 1)\norm{x},
\end{equation}
where $\epsilon$ is maximal relative backward error in application of any
$\wt Q_1,\dots,\wt Q_r$, and $\wh Q_i$ is orthogonal matrix close to $\wt
Q_i$.

\begin{proposition}\label{prop:qr-errbd}
  Let the Givens or Householder QR factorisation 
  $
  A = Q
  \begin{pmatrix}
    R \\ 0
  \end{pmatrix}
  $
  of $A\in\R\mn$, $m\ge n$, be computed in the IEEE floating point
  arithmetic with rounding relative error $\varepsilon < 10^{-7}$. Let the
  computed approximations of $Q$ and $R$ be $\wt Q$ and $\wt R$,
  respectively. Then, there exist an orthogonal matrix $\wt Q$ and a
  backward perturbation $\delta A$ such that $A + \delta A = \wt Q
  \begin{pmatrix}
    \wt R \\ 0
  \end{pmatrix}
  $, where $\fnorm{\wt Q(:,i) - \wh Q(:,i)} \le \varepsilon_{qr}$ and 
  $\norm{\delta A(:,i)} \leq \varepsilon_{qr}\norm{A(:,i)}$, $1\le i\le n$, 
  hold with  
  \begin{enumerate}
\item
  $\varepsilon_{qr}\le O(mn)\varepsilon$ for the Householder QR
  factorisation.  
\item 
  $\varepsilon_{qr}\le (1+6\varepsilon)^p - 1$ for the Givens
  factorisation. 
  \end{enumerate}
\end{proposition}

\subsection{Condition Number and Scaling}\label{sec:cond-numb-scal}

It is often that the matrix is composed as $A = BD$, $D =
\diag(d_i)_{i=1}^n$, and the uncertainty is $A + \delta A = (B + \delta
B)D$. In such case, $\delta A A ^{\dagger} = \delta BB^{\dagger}$.
Moreover, the scaling, diagonal matrix $D$, which might be ill-conditioned,
does not influence the forward perturbation of the singular values in
(\ref{eq:fwd-error-sv}), i.e. 
\begin{equation}\notag
  \max_{i=1:n}\frac{|\wt \sigma_i-\sigma_i|}{\sigma_i} \leq 
\norm{\delta A A^{\dagger}} = \norm{\delta BB^{\dagger}}.
\end{equation}

Similarly, if $A = DB$ (might be row scaling, instead of previous column
scaling), and the uncertainty is $A + \delta A = D(B + \delta B)$, then we
have the bound can also reduced to $B$ and $\delta B$. Hence, no matter how
$A$ is ill-conditioned, as long as the condition number of the scaled $A$
is low, we can have a sharp bound.

Therefore, after we compute the QR factorisation of an $m\times n$ full
column rank matrix $A$, we can conclude that the backward perturbation
satisfies 
\begin{equation}\notag
  \norm{\delta AA^{\dagger}} = \norm{(\delta AD^{-1})(AD^{-1})^{\dagger}}
  \leq  \norm{\delta AD^{-1}}\norm{(AD^{-1})^{\dagger}}\leq
  \sqrt{n}\epsilon_{qr} \norm{A_c^{\dagger}},
\end{equation}
where $D = \diag(\norm{A(:,i)})_{i=1}^n$ and $A_c = AD^{-1}$. Moreover, we
have the following proposition
\begin{proposition}
  Let $A$ and $\wt R$ be as in Proposition \ref{prop:qr-errbd}. If
  $\sigma_1\ge \cdots  \ge \sigma_n > 0$ and $\wt \sigma_1 \ge \cdots \ge
  \wt \sigma_n > 0$ are the singular values of $A$ and $\wt R$,
  respectively, then  
  \begin{equation}\notag
    \max_{i = 1:n}\frac{|\wt \sigma_i-\sigma|}{\sigma_i} \leq  
    \sqrt{n}\varepsilon_{qr}\norm{A_c^{\dagger}} \le n \varepsilon_{qr} 
    \min_{\substack{S = \diag \\ \det(S)\neq 0}} \kappa(AS). 
  \end{equation}
\end{proposition}

\emph{Conclusion: If $A$ is such that $\kappa(AS)$ is moderate for some
  diagonal matrix $S$, then floating point $QR$ factorisation preserves all
singular values -- they are all safely passed to the computed triangular
factor $R$.} 

Also note that, $R_c = RD^{-1}$ has unit columns and $\norm{R_c^{-1}} =
\norm{A^{\dagger}_c}$. The note can stopped here, since all we need to know
is that: doing QR factorizations, as long as the scaled $A$ is not too
wild, then we can safely computing the SVD of $R$, its upper triangular QR
factor, instead of the SVD of $A$.

\section{Preconditioned Jacobi SVD algorithm}
\label{sec:prec-jacobi-svd} 

In case of $m \gg n$, the QR factorization of $A$, $A = Q
\begin{pmatrix}
  R \\ 0
\end{pmatrix}
$ reduces the computational complexity of all classical SVD method.

\emph{The big picture from Bidiagonalization SVD.}
Bidiagonalization based SVD algorithms reduce $A$ or $R$ to bidiagonal form
in $4mn^{2} - 4n^3/3$ flops. Recent implementation of bidiagonalization
substantially reduces the data transfer between main memory and cache. Thus
iterative part of those algorithms runs on a bidiagonal matrix, and
completes by assembling the orthogonal matrix QR factorization
bidiagonalization from the bidiagonal SVD. If we add the fact that the full
SVD of bidiagonal matrix can be computed very efficiently, then the picture
is complete. \bf{Efficient preprocessing reduces the problem to the one
  with super fast and ingenious solution. Efficient postprocessing
  assembles all elements of the SVD.}

Therefore, for Jacobi SVD algorithm, we can transform the problem from $m\times
n$ matrix to $n\times n$. In the case of $m\gg n$, the QR factorization is an
attractive preprocessor. Since the most expensive iterative part of the
Jacobi SVD algorithm transforms $n$-dimensional vectors instead of
$m$-dimensional one.

\subsection{The QR factorisation as preprocessor}
Using QR factorisation as efficient preprocessor for the Jacobi SVD routine
is more subtle for several reasons:
\begin{enumerate}
  \item First, the matrix $R$ is not only smaller than $A$, in case $m >
    n$, but it is also triangular which allows additional savings. For
    instance, if we partition $R$ as 
    \begin{equation}\notag
      R =
      \begin{pmatrix}
        R_{[11]} & R_{[12]}\\ 0 & R_{[22]}
      \end{pmatrix},
    \end{equation}
    where $R_{[11]}$ is $k\times k$, $k = \lfloor n/2 \rfloor$, then during
    the first sweep $k(k-1)/2$ rotations of the columns $R_{[11]}$ can be
    performed in a canonical $k$-dimensional subspace.
  \item Second, since the Jacobi algorithm iterates on the full matrix, we
    are interested not only in preprocessing (in the sense of dimensional
    reduction described above), but also in preconditioning in the sense of
    inducing fast convergence. This raises the question of \emph{How to
      use QR factorization(s) to precondition the initial matrix $A$?}
  \item Let $R = U_R\Sigma V_R^T$ be the SVD of $R$, where $V_R$ is the
    (infinite) product of Jacobi rotations, and let both sets of singular
    vectors be required ($U$ and $V$). If $R$ is nonsingular, then $V_R =
    R^{-1}U_R\Sigma$. It is tempting to implement Jacobi algorithm without
    accumulation of Jacobi rotations and to compute $V_R$ from triangular
    matrix equation using BLAS 3 operation
    \href{https://netlib.org/lapack/explore-html/db/dc9/group__single__blas__level3_ga9893cceb3ffc7ce400eee405970191b3.html}
    {\texttt{STRSM}}
    A fast rotation for two $n\times 1$ vectors has $4n$ flops, one sweep of
    $n(n-1)/2$ rotations has $2n^3 - 2n^2$ BLAS 1 flops, while
    \texttt{STRSM} has $n^3$ BLAS 3 flops. 
\end{enumerate}

\emph{How can the QR factorization serve as a preconditioner for better,
  faster, convergence of the Jacobi algorithm?} This is achieved if the
factorization is computed with column pivoting of Businger and
Golub~\cite{bugo65}:
\begin{equation}\label{eq:qr-col-pivoting}
  AP = Q 
  \begin{pmatrix}
    R \\ 0
  \end{pmatrix},
\quad 
\text{$P$ permutation such that }
\abs{R_{ii}} \ge \sum_{k=i}^jR_{kj}^2, \quad 1 \le i < j \le n.
\end{equation}

Now, note that SVD of $R$ is implicit diagonalization of the matrix
$R^TR$, and apply one-step of Rutishauser LR diagonalization method $R^TR
\implies RR^T$ (\comment{may be like QR algorithm?}), which has
a nontrivial diagonalizing effect. 

This means, $RR^T$ is closer to diagonal form than $R^TR$. Note that 
\begin{equation}\notag
  \begin{pmatrix}
    RR^T & 0 \\ 0 & 0 
  \end{pmatrix} = 
  Q^T(AA^T)Q, \qquad R^TR = P^T(A^TA)P,
\end{equation}
these two orthogonal similarities are substantially different.

In terms of the Jacobi algorithm, it will translates to: \emph{One-sided
  Jacobi SVD algorithm on $R^T$ should have better convergence than applied
to $R$.} Therefore, by Veseli{\'c} and Hari~\ycite[1989, p.~631]{veha89},
we have the following preconditioning method: Given a general $A\in\R\mn$,
\begin{equation}\notag
  AP = Q 
  \begin{pmatrix}
    R \\ 0
  \end{pmatrix},
  \qquad R^T = Q_1R_1,\qquad R_1^TP_1 = Q_2R_2.
\end{equation}

Moreover, it is known that \emph{simply sorting the columns of $A$ in
  non-increasing order (in Euclidean norm) improves the convergence.} The
best elaborated pivoting is the one by de Rijk~\cite[Sec.~2.4]{deri89}.

The following proposition shows how the pivoting (\ref{eq:qr-col-pivoting})
influences the condition number of the row-scaled matrix $R$.
\begin{proposition}\label{fwd-sens-diag-scaling}
  Let $R$ be a nonsingular upper triangular matrix and let
  (\ref{eq:qr-col-pivoting}) hold. Let $R = \Delta_RR_r$, where $\Delta_R$
  is the diagonal matrix of the Euclidean lengths of the rows of $R$, and
  let $R = R_cD_R$, $D_R = \diag(\tnorm{R(:,i)})$. Then 
  \begin{equation}\notag
    \begin{aligned}
      \tnorm{\abs{R_r^{-1}}} & \leq \sqrt{n} +
                               \max_{i<j}\frac{(\Delta_R)_{jj}}{(D_R)_{ii}}
                               \cdot \tnorm{\abs{R_c^{-1} -
                               \diag(R_c^{-1})}} \\
                            &  \leq \sqrt{n}
                              \left (1+\max_{i<j}\frac{\abs{R_{jj}}}{\abs{R_{ii}}}
                              \cdot \tnorm{\abs{R_c^{-1} -
                              \diag(R_c^{-1})}}\right), \\
      \tnorm{\abs{R_r^{-1}}} & \leq \sqrt{n} \tnorm{\abs{R_c^{-1}}} 
    \end{aligned}
  \end{equation}
  where the matrix absolute value is defined element-wise. Moreover
  $\norm{R_r^{-1}}$ is bounded by a function of $n$, independent of $A$.
\end{proposition}

\begin{remark}
  Our understanding of the rank revealing property of the QR factorization
  is different. \emph{If upper triangular factor in the column pivoted QR
    factorization is written as $R = DY$ with diagonal $D$, where $D =
    \diag(\abs{R_{ii}})_{i=1}^n$ or $D = \diag(\norm{R(i,:)})_{i=1}^n,
    D_{ii} \geq D_{i+1,i=1}$, then determine the pivoting to minimize
    $\kappa(Y) = \norm{Y^{-1}}\norm{Y}.$}
\end{remark}

Since the choice of $D$ and the diagonal dominance ensure that $\norm{Y}$
is bounded by $O(n)$, the problem reduces to \emph{Minimising
  $\norm{Y^{-1}}$.}
Gu and Eisenstat has theoretical bound for $\norm{Y\inv}$ which is
comparable to the Wilkinson's pivot growth factor $O(n^{\log_2n/4})$ in the
Gaussian elimination with complete pivoting. The bound in case of the
Businger-Golub pivoting is exponential in $n$. However, in practice,
$\tnorm{Y\inv}$ is typically bounded by $O(n)$.

\begin{remark}
  If $m \gg n$, then the cost of the RRQR dominates the overall complexity
  of the SVD computation. Since the pivoting can slow down the QR
  factorization by a significant amount. \emph{It is reasonable that in
    case $m\gg n$, the computation starts with the QR factorization without
  pivoting just to reduce the dimension.}
\end{remark}

\section{Simple Question: $A$ or $A\tp$?}
For the SVD of $A\tp$ is trivial to obtain from the SVD of $A$ and vice
versa. If $A$ is $m\times n$ with $m > n$, then we prefer starting the
computation with the RRQR of $A$. If $m < n$, then we choose to start with
$A\tp$. But what if $A$ is square non-singular $n\times n$ matrix? 

Consider an extreme example: $A = DQ$, where $D$ is diagonal and $Q$ is
orthogonal. In that case, working with $A$ is implicit diagonalization of
$Q\tp D^2Q$, while taking $A\tp$ implicitly diagonalizes diagonal matrix
$D^2$. Therefore, we would like to find out \emph{which of the matrices
  $A\tp A$ and $AA\tp$ is closer to the diagonality?} This is particularly
useful when $A$ is not normal.

Due to computational constraints, we only allow to make a decision at most
$O(n^2)$ flops. The complexity corresponding to \emph{computing the
  diagonal entries of $H = A\tp A$ and $M = AA\tp$.} Notice that $M$ and
$H$ are orthogonally similar, therefore $\fnorm{H} = \fnorm{M}$. Moreover,
we can compute the values $s(H) = \sum_{i=1}^nh_{ii} = \tr(H\circ H)$ and
$s(M)$. \emph{Larger value of $s(\cdot)$ corresponding to smaller
  $\mathrm{off}(\cdot)$.} 
However, $s(\cdot)$ in floating point computation with round-off
$\varepsilon$ completely ignores diagonal entries below
$\sqrt{\varepsilon}$.

\section{The Algorithms}
Now, describe the structure of the Jacobi SVD algorithm with QR
factorization as preconditioner and preprocessor. We consider the one-sided
Jacobi as the following ``box'':
\begin{equation}\notag
  \text{Full column rank $X$} \implies \boxed{\text{One-sided Jacobi}}
  \implies X_{\infty} = XV.
\end{equation}
The notation $\inner{\cdot}$ indicate the matrix that is not computed and
no information about it is stored. The algorithms are designed such that
avoiding accumulation of Jacobi rotations whenever possible.

\subsection{Computing only $\Sigma$}

In Algorithm~\ref{alg:only-sv}, we use two QR factorizations with pivoting
and then apply the one-sided Jacobi SVD algorithm.
\begin{algorithm}
\caption{$\sigma = \texttt{svd}(A)$}
\label{alg:only-sv}
  \begin{algorithmic}[1]
    \State {$(P_rA)P = \inner{Q}
      \begin{pmatrix}
        R \\ 0
      \end{pmatrix}; \quad \rho = \rank(R);$}
    \State $R(1:\rho,1:n)^TP_1 = \inner{Q_1}R_1;$
    \State $X = R_1\tp; X_{\infty} = X\inner{V_x};$
    \State $\sigma_i = \norm{X_{\infty}(:,i)}, \quad i = 1,\dots,\rho;$
    \State \boxed{\sigma = (\sigma_{1},\dots, \sigma_p,0,\dots, 0).}
  \end{algorithmic}
\end{algorithm}

\begin{remark}
  The pivoting in the second QRF is optional, and $P_1 = I$ works well. If
  the columns of $A$ are nearly orthogonal, the second QR factorization is
  unnecessary. Such situation is easily detected by inspecting the matrix
  $R$, see~\cite{drve08ii}.
\end{remark}

\begin{remark}
  Determining the numerical rank is tricky. Since if high accuracy is
  required, then QR factorisation is not authorised to declare rank
  deficiency. 
\end{remark}

\subsection{Computing $\Sigma$ and $V$}
If we need \emph{singular values and the right singular vectors}. Direct
application of right-handed Jacobi rotation to $A$ or $R$ requires the
accumulated product of rotation. To avoid this explicit multiplication, we
consider $R \rightsquigarrow R^T$.

\begin{algorithm}
  \caption{$(\sigma, V) = \texttt{svd}(A)$}
  \begin{algorithmic}[1]
    \State $(P_rA)P = \inner{Q} \begin{pmatrix} R \\ 0 \end{pmatrix}; \quad
    \rho = \rank(R);$ 
    \State $X = R(1:\rho,1:n)^T; X_{\infty} = X\inner{V_x}$;
\State $\sigma_i = \norm{X_{\infty}(:,i)}, \quad i = 1,\dots,\rho; \quad 
\boxed{\sigma = (\sigma_1,\dots,\sigma_\rho, 0,\dots, 0)};$
\State $U_x(:,i) = \frac{1}{\sigma_i}X_{\infty}(:,i),\quad i =
1,\dots,\rho;
\quad \boxed{V = PU_x}$
  \end{algorithmic}
\end{algorithm}
Notice that, we only have one QR factorization appeared. In some cases,
such as $\rho \ll n$, the second QR factorization is advisable.

\subsection{Computing $\Sigma$ and $U$} 
If $\Sigma$ and $U$ are needed, we need to think harder. Clearly, if we
apply the right handed Jacobi on $X=A$ or $X=R$, then we do not need the
product of Jacobi rotations. The problem is that \emph{in case $m\gg n$,
  the rotations on $A$ are too expensive and the convergence may be slow,
  much slower than in the case $X=R^T$.}

On the other hand, in some cases $X=A$ is the perfect choice. For instance,
if $H$ is symmetric positive definite, and we have the pivoted Cholesky
fractorization $P^THP = AA^T$ with lower triangular matrix $A$, then $A^T$
has the same properties as $R$ from Proposition
\ref{fwd-sens-diag-scaling}. Thus $AV = U\Sigma$ will be efficient Jacobi
SVD and since $H=(PU)\Sigma^2(PU)\tp$ is the spectral decomposition of $H$
and $V$ is not needed.  

To simplify the notation, in Algorithm~\ref{alg:svd-sv-lhsvec}, we define
for a matrix $M$ its property $\tau(M)$ to be true if $M$ is of full column
rank and the right-handed Jacobi algorithm applied to $M$ converges
quickly. Such as the Cholesky factor of positive definite matrix, computed
with pivoting, then $\tau(A) = \mathrm{true}$. If evaluation of $\tau(A)$
would require more than $O(mn)$ flops, or if we do not know how to judge
$A$, then by definition $\tau(A) = \mathrm{false}$.

\begin{algorithm}
  \caption{$(\sigma,U) = \texttt{svd}(A)$} \label{alg:svd-sv-lhsvec}
  \begin{algorithmic}[1]
\If {$\tau(A)$}
    \State $X = A; X_{\infty} = X\inner{V_x};$
    \State $\sigma_i = \norm{X_{\infty}(:,i)}, \quad i = 1,\dots,n,\quad 
    \boxed{\sigma = (\sigma_1,\dots,\sigma_n};$
\Else
    \State $(P_rA)P = \inner{Q} \begin{pmatrix} R \\ 0 \end{pmatrix}; \quad
    \rho = \rank(R);$  
    \If {$\tau(R)$}
        \State $X = R; X_{\infty} = X\inner{V_x};$
        \State $\sigma_i = \norm{X_{\infty}(:,i)}, \quad i = 1,\dots,\rho; 
\boxed{\sigma = (\sigma_1,\dots,\sigma_p,0,\dots,0)};$ 
        \State $U_x(:,i) = \frac{1}{\sigma_i}X_{\infty}(:,i),\quad i =
        1,\dots,\rho; \quad \boxed{U = P_r\tp Q
        \begin{pmatrix}
          U_x \\ 0_{(m-\rho) \times \rho}
        \end{pmatrix}};$
    \Else 
        \State $R(1:\rho,1:n)^TP_1 = \inner{Q_1}R_1$;
        \State $X = R_1^T; X_{\infty} = X\inner{V_x}$;
        \State $\sigma_i = \norm{X_{\infty}(:,i)}, \quad i = 1,\dots,\rho; 
\boxed{\sigma = (\sigma_1,\dots,\sigma_p,0,\dots,0)};$ 
        \State $U_x(:,i) = \frac{1}{\sigma_i}X_{\infty}(:,i),\quad i =
        1,\dots,\rho; \quad \boxed{U = P_r\tp Q
        \begin{pmatrix}
          P_1U_x \\ 0_{(m-\rho) \times \rho}
        \end{pmatrix}};$
    \EndIf
\EndIf
  \end{algorithmic}
\end{algorithm}
 In the case, $X = R\tp$, we need the accumulated product of Jacobi
 rotations and the cost for only one sweep of rotations is $2n\rho(\rho -
 1)$. All of this is avoid by an extra QR factorisation.

\subsection{Computation of $U$, $\Sigma$ and $V$}

Classical implementation of the Jacobi SVD algorithm transforms two
matrices, one of them approaching the matrix of left singular vectors
scaled by the corresponding singular values, and the second one is the
accumulated product of the Jacobi rotations.

Following~\cite{drma99}, we will try not to accumulate Jacobi rotations,and
the right singular vectors will be computed a posterior from a
\emph{well-conditioned matrix equation}.

\subsubsection{Classical computation of $V$ by accumulation}
Let the Jacobi iterations stop at index $\wb k$ and let $\wt X_{\infty} =
\wt X^{(\wb k)}$. Let $\wt V$ be the computed accumulated product of Jacobi
rotations used to compute $\wt X_{\infty}$. Row-wise backward stability
implies 
\begin{equation}\notag
  \wt X_{\infty} = (X + \delta X)\wh V
\end{equation}
where $\wh V$ is orthogonal, $\norm{\wh V - \wt V} \leq O(n\epsilon)$ and
$\norm{\delta X(i,:)} \leq \epsilon_J\norm{X(i,:)},\ \epsilon_J\leq
O(n\varepsilon)$. The matrix $\wt V$ can be written as $\wt V = (I +
E_0)\wh V$, where $\norm{E_0}$ is small. Therefore, to recover $\wh V$, the
best we can do is 
\begin{equation}\notag
  X\inv \wt X_{\infty} = (I + E_1)\wt V, \quad E_1 = X\inv \delta X.
\end{equation} 
Although we do not have $\delta X$, we can let $E_1$ close to zero. To
estimate $E_1$, we write $X = DY$, where $D$ is diagonal scaling, $D_{ii} =
\norm{X(i,:)}$, and $Y$ has unit rows in Euclidean norm. We obtain
\begin{equation}\notag
  \norm{E_1} = \norm{Y\inv D\inv \delta X} \leq \norm{Y\inv} \norm{D\inv
    \delta X} \leq \norm{Y\inv}\sqrt{n}\epsilon_J \leq \norm{Y\inv}
  O(n^{3/2}\varepsilon).
\end{equation}
Therefore, as long as the $X$'s row scaled part is well-conditioned, we can
have 
\begin{equation}\notag
  \frac{\norm{X^{-1}\wt X_{\infty} - \wh V}}{\norm{\wh V}} \leq \norm{E}
\end{equation}

Finally, the matrix $\wt X_{\infty}$ is written as $\wt U \wt \Sigma$. The
diagonal entries of $\wt \Sigma$ are computed as $\wt \sigma_i =
computed(\norm{\wt X_{\infty}(:,i)}) = \norm{\wt
  X_{\infty}(:,i)}(1+\nu_i),\ \abs{\nu_i} \leq O(n \varepsilon),$ and then
$\wt U(:,i)$ is computed by dividing $\wt X_{\infty}(:,i)$ by $\wt
\sigma_i$, thus 
\begin{equation}\notag
  \wt U \wt \Sigma = \wt X_{\infty} + \delta\wt X_{\infty},\qquad 
\abs{\delta \wt X_{\infty}} \leq 3 \varepsilon \abs{\wt X_{\infty}}.
\end{equation}

The following proposition explains how well the computed SVD resembles the
matrix $X$.

\begin{proposition} \label{prop:svd-residual-err}
  The matrix $\wt U, \wt \Sigma, \wt V$ and $\wh V$ satisfy the residual
  relations 
\begin{equation}\notag
  \begin{aligned}
    \wt U \wt \Sigma \wh V^T& = X + F \\
    \wt U \wt \Sigma \wt V\tp & = (X + F)(I + E_0\tp),
  \end{aligned}
\end{equation}
where for all $i$, 
\begin{align*}
  \norm{F(i,:)} & \leq (\varepsilon_J + 3\varepsilon(1 +
                  \varepsilon_J))\norm{X(i,:)}, \\
\norm{E_0} & \leq \sqrt{n}\varepsilon_J \leq O(n^{3/2}\varepsilon),\\
\norm{X\inv F} & \leq \norm{Y\inv} \sqrt{n} (\varepsilon_J + 3\varepsilon(1+\varepsilon_J)).
\end{align*}
\end{proposition}

\subsubsection{Computation of $V$ from matrix equation}

Suppose we decide to use an approximation of $\wh V$ instead of $\wt V$.
The matrix $X\inv \wt X_{\infty}$ is a good candidate, but we cannot have
the exact value of $X\inv \wt X_{\infty}$. Instead, \emph{we solve the
  matrix equation $\wt X_{\infty} = X\breve{V}$\footnote{This can be solved
  using BLAS 3 \texttt{STRSM}.} and take $\breve{V} = computed(X\inv \wt X_{\infty})$}

Since $X$ is triangular, the residual bound for $\breve{V}$ is 
\begin{equation}\notag
  E_2 = X\breve{V} - \wt X_{\infty}, \quad \abs{E_2} \leq
  \epsilon_T\abs{X}\abs{\breve{V}}, \quad \epsilon_T \leq
  \frac{n\varepsilon}{1- n\varepsilon}. 
\end{equation}

The following proposition shows that we have also computed a rank revealing
decomposition of $X$,
\begin{proposition}
  The matrix $\wt U, \wt \Sigma, \breve{V}$ satisfy the following residual
  relations 
  \begin{equation}\notag
    \begin{aligned}
          \wt U \wt \Sigma \breve V\tp & = (X + F)(I + E_3^T), \quad 
    E_3 = E_1 + X\inv E_2 \wt V^T,\\
\wt U \wt \Sigma \breve V\inv & 
= X+ F_1, \quad 
F_1 = E_2\breve V\inv + \delta \wt X_{\infty} \breve{V}\inv,
    \end{aligned}
  \end{equation}
where $F$ is in Proposition~\ref{prop:svd-residual-err}, $\norm{E_3} \leq
\norm{Y\inv}(\sqrt{n}\epsilon_J + n\epsilon_T)$, and it holds for all $i$, 
$\norm{F_1(i,:)} \leq (\varepsilon_T\norm{\ \abs{\breve{V}}\ } + 3\varepsilon_J(1+\varepsilon_J))\norm{\breve{V}\inv}\norm{X(i,:)}$
\end{proposition}

Analysis shows that the quality of the computed right singular vector
matrix $\breve V$ depends on the condition number of $Y\inv$, where $X =
DY$. This means, the rows of the triangular $X$ must be well-conditioned in
the scaled sense. If $X$ is computed from the initial $A$ using the QR
factorization with column pivoting, $AP = Q
\begin{pmatrix}
  R \\ 0
\end{pmatrix}
$, then $X = R$ can be written as $X = DY$ with well-conditioned $Y$. Thus,
we expect that $\breve{V}$ can be computed accurately.

Conclusion: The initial matrix should be of the form $X = DY = ZC$ where
$D,C$ are diagonal and both $Y$ and $Z$ well-conditioned. Well-conditioned
$Z$ implies fast convergence, while well-conditioned $Y$ ensures a stable
posterior computation of the right singular vectors. Therefore, we define
$X$ in the following way:
\begin{equation}\notag
  AP = Q
\begin{pmatrix}
  R \\ 0
\end{pmatrix}; \quad R\tp P_1 = Q_1R_1; \quad X = R_1^T.
\end{equation}




\section{Assessing the accuracy of the computed SVD (Skip for now)}



%%%%%%%%%%%%%%%%%%%%%%%%%%% Bibliographies %%%%%%%%%%%%%%%%%%%%%%%%%%%
\newpage 
\bibliographystyle{/Users/clement/bibtex/nj-plain}
\bibliography{/Users/clement/bibtex/references.bib}
%% APPENDICES
\end{document}

%%% Local Variables:
%%% mode: latex
%%% TeX-master: t
%%% End:
