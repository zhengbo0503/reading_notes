\documentclass{article}
\def\ntitle {Note : \texttt{high08\_FM} Chapter 6}
% \def\nauthor{ } % default author is Zhengbo Zhou
% \def\notes{ } % default is the submitted version
% \def\ndate{ } % default is today's date
\def\needcrop{ } % crop for easy previewing
\def\fancysec{ } % section font become fancier 
%%%%%%%%%%%%%%%%%%%%%%%%%%%%%%%%%%%%%%%%%%%%%%%%%%%%%%%%%%%%%%%%%%%%%
%%%%%%%%%%%%%%%%%%%%%%%% Systematic Settings %%%%%%%%%%%%%%%%%%%%%%%%
%%%%%%%%%%%%%%%%%%%%%%%%%%%%%%%%%%%%%%%%%%%%%%%%%%%%%%%%%%%%%%%%%%%%%

% DEFINE ntitle/nauthor/notes/ndate
\RequirePackage{etex}
\makeatletter
\ifx \nauthor\undefined
  \def\nauthor{Zhengbo Zhou}
\else 
\fi 
\ifx \ndate\undefined 
  \def\ndate{\today}
\else 
\fi 
\ifx \notes\undefined 
\else
    \makeatletter
    \renewcommand\maketitle
        {\noindent
            {\large\scshape\ntitle}
            \medskip\par\noindent
            {\slshape\nauthor}
            \hfill
            {\normalsize\itshape\ndate}
            \bigskip\par\noindent
        }
    \makeatother
\fi

\author{\nauthor \thanks{
Department of Mathematics,
University of Manchester,
Manchester, M13 9PL, England
(\texttt{zhengbo.zhou@postgrad.manchester.ac.uk}).
}}
\date{\ndate}
\title{\ntitle}

%% Required Packages
\usepackage{algorithm}
\usepackage{algpseudocode}
\usepackage{comment}
\usepackage{bookmark}
\usepackage{microtype}
\usepackage{booktabs}
\usepackage{lastpage}
\usepackage{fancyhdr}
\usepackage{amsthm}
\usepackage{mathtools}
\usepackage{enumerate}
\usepackage{mathrsfs}
\usepackage{amsfonts}
\usepackage{amssymb}
\usepackage[theorems]{tcolorbox}
\usepackage{bm}
\usepackage{cancel}
\usepackage{bbm}
\usepackage{accsupp}
\usepackage{enumitem}
\usepackage{hyperref}
\usepackage[fontsize=12pt]{fontsize}
\usepackage{geometry}
\usepackage{microtype}
\usepackage{anyfontsize}
\usepackage[hyperpageref]{backref}

%% Avoid badscalings
\renewcommand\hmmax{0}
\renewcommand\bmmax{0}
\hbadness=99999

%% page size
\geometry{
  a4paper,
  textwidth=159truemm,
  textheight=240truemm,
  top=24.5truemm,
}

%% table of content settings
\setcounter{tocdepth}{2}

%% hyperref and backref
\hypersetup{
    hypertexnames=false, 
    colorlinks,
    linkcolor={red!50!black},
    citecolor={green!50!black},
    urlcolor={blue!80!black},
    pdfauthor={Zhengbo Zhou},
    pdftitle={\ntitle},
    pdfcreator={Zhengbo Zhou via MikTeX}
}
\renewcommand*{\backref}[1]{}
\renewcommand*{\backrefalt}[4]{
    \ifcase #1 
    No citations.
    \or
    (Cited on p.~#2.)
    \else
    (Cited on pp.~#2.)
    \fi
}

%% header and footer
\fancypagestyle{plain}{
    \fancyhf{}
    \renewcommand{\headrulewidth}{0pt}
    \lhead{\textit{\nouppercase{\leftmark}}}
    \rhead{\ntitle}
    \cfoot{\thepage\ of \pageref*{LastPage}}
}
\fancypagestyle{firstpage}[plain]{
    \renewcommand{\headrulewidth}{0pt}
    \fancyhead[L,C,R]{}
    \fancyfoot[L,C,R]{}
}
\pagestyle{plain}

%% colorboxes
\tcbuselibrary{breakable}
\newtcolorbox{mycolorbox}[1]{ % grey background box
    colback=white!90!black,
    colframe=white!40!black,
    fonttitle=\scshape\centering,
    title=#1
}
\newtcolorbox{mybox}[1]{ % pure white box with border
    colback=white,
    colframe=black,
    boxrule=0.75pt,
    breakable,
    arc=0pt,
    outer arc=0pt,
    fonttitle=\scshape\bfseries
} 

%% redefine \emph and \bf such that they are colored and can be easily
%% transformed back to normal \emph and \bf
\renewcommand{\emph}[1]{\textit{\color{purple} #1}}
\renewcommand{\bf}[1]{\textsf{\bfseries \color{purple} #1}}

%% Theorems 
\newtheorem{theorem}{Theorem}[section] % Italic font 
\newtheorem{proposition}[theorem]{Proposition}
\newtheorem{corollary}[theorem]{Corollary}
\newtheorem{lemma}[theorem]{Lemma}

\theoremstyle{definition}
\newtheorem{definition}[theorem]{Definition} % upright font
\newtheorem{example}[theorem]{Example}
\newtheorem{remark}[theorem]{Remark}
\newtheorem*{question}{Question}
\newtheorem*{recall}{Recall}
\newtheorem*{assumption}{Assumption}
\newtheorem*{note}{Note}

\numberwithin{equation}{section} % equation numbers

%%%%%%%%%%%%%%%%%%%%%%%%%%%%%%%%%%%%%%%%%%%%%%%%%%%%%%%%%%%%%%%%%%%%%
%%%%%%%%%%%%%%%%%%%%%%%% USER DEFINED MACROS %%%%%%%%%%%%%%%%%%%%%%%%
%%%%%%%%%%%%%%%%%%%%%%%%%%%%%%%%%%%%%%%%%%%%%%%%%%%%%%%%%%%%%%%%%%%%%

%% MATHBB
\newcommand{\mb}[1]{\mathbb{#1}}
\newcommand{\N}{\mb{N}}
\newcommand{\Z}{\mb{Z}} 
\newcommand{\Q}{\mb{Q}}
\newcommand{\R}{\mb{R}} 
\newcommand{\C}{\mb{C}}
\newcommand{\F}{\mb{F}}
\renewcommand{\P}{\mb{P}} %% Probability
\newcommand{\E}{\mb{E}} %% Expectation
\newcommand{\V}{\mb{V}} %% Variance

%%%%%% MATHCAL and MATHSCR
\newcommand{\mc}[1]{\mathcal{#1}} %% For spaces 
\newcommand{\ms}[1]{\mathscr{#1}} %% For sigma-algebra

%% Paired labels
\DeclarePairedDelimiter\ceil{\lceil}{\rceil} % ceiling 
\DeclarePairedDelimiter\floor{\lfloor}{\rfloor}  % floors
\DeclarePairedDelimiter\inner{\langle}{\rangle} %% <a>
\DeclarePairedDelimiter\abs{\lvert}{\rvert} %% |a|
\makeatletter
\let\oldabs\abs
\def\abs{\@ifstar{\oldabs}{\oldabs*}}

%% vector bold
\renewcommand{\vec}[1]{\bm{#1}}

%% Real and Imaginary
\renewcommand{\Re}[1]{\mathrm{Re}\left\{{#1}\right\}}
\renewcommand{\Im}[1]{\mathrm{Im}\left\{{#1}\right\}}

%% Integrate from ... to ...
\newcommand{\intii}{\int_{-\infty}^{\infty}}

%% Greek Letters
% NEVER define \l for \lambda due to Polish names in BibTeX
\newcommand{\eps}{\varepsilon}

%% Matrix Analysis
\newcommand{\Rn}{\R^n}
\newcommand{\Cn}{\C^n}
\newcommand{\n}{^{n}}
\newcommand{\m}{^{m}}
\newcommand{\mn}{^{m\times n}}
\newcommand{\nn}{^{n\times n}}
\newcommand{\tp}{^{T}} % transpose 
\newcommand{\ctp}{^{*}} % conjugate transpose
\newcommand{\inv}{^{-1}}
\DeclareMathOperator{\diag}{diag}
\DeclareMathOperator{\rank}{rank}
\DeclareMathOperator{\tr}{trace}
\DeclareMathOperator{\range}{Range}
\newcommand{\iter}[1]{^{(#1)}} %% iteration \iter{}

%% Norms
\newcommand{\norm}[1]{\|{#1}\|}
\newcommand{\fnorm}[1]{\norm{#1}_F}
\newcommand{\tnorm}[1]{\norm{#1}_2} % 2-norm
\newcommand{\inorm}[1]{\norm{#1}_\infty} % infinity norm

%% Over the expressions
\newcommand{\wh}{\widehat}
\newcommand{\wt}{\widetilde}
\newcommand{\wb}{\overline}

%% Calculus
\newcommand{\lap}{\varDelta}
\DeclareMathOperator{\grad}{\nabla}
\renewcommand{\div}{\nabla\cdot}
\DeclareMathOperator{\curl}{\nabla\times}
\newcommand{\dd}{\mathrm{d}}
\newcommand{\pp}{\partial}
\def\eu{\mathrm{e}} % euler's constant
\def\im{\mathrm{i}} % imaginary unit
\DeclareMathVersion{bold}
\newcommand{\indi}{\mathbbm{1}} %% indicator function

%% Citation % nick
\def\ycite[#1#2#3#4#5]#6{\cite[$\mit{#1#2#3#4}$#5]{#6}}

\usepackage{anyfontsize}
\usepackage{newtxtext}     
\usepackage[varvw]{newtxmath} 
\def\ttdefault{cmtt}
% \usepackage[ bb=ams ]{mathalpha}

\makeatother



\newcommand{\textus}[1]{\textup{\textsf{#1}}}
\renewcommand{\comment}[1]{\textus{\color{violet}\ $\heartsuit$(#1)}}
\newcommand{\sign}{\mathrm{sign}}


\begin{document}
\maketitle
\tableofcontents
\thispagestyle{firstpage} 
%%%%%%%%%%%%%%%%%%%%%%%%%%%% MAIN ARTICLE %%%%%%%%%%%%%%%%%%%%%%%%%%%%
\begin{abstract}
  This is a note for \cite{high08_FM}, and this file by no mean serve as a
  correct reference for knowledge. This is only for myself convenience. 
\end{abstract}

\section{Introduction}\label{sec:introduction}
In this chapter, we will always consider the principal square root, denoted
as $A^{1/2}$. Recall that if $A\in\C\nn$ with no eigenvalues on $\R^-$,
$A^{1/2}$ is the unique square root $X$ of $A$ whose spectrum lies in the
open right half-plane. We will denote $\sqrt{A}$ be any arbitrary, possibly
non-principal square roots. The matrix (principal) square root also
has an integral expression:
\begin{equation}\label{eq:sqrt-integral}
  A^{1/2} = \frac{2}{\pi}A\int_0^{\infty}(t^2I + A)^{-1}~\dd t.
\end{equation}

The layout of this chapter will be 
\begin{enumerate}
\item Sensitivity and conditioning: Analysis of the conditioning of the
  matrix square root, and the sensitivity of the relative residual.
\item Schur method: A Schur method and a version working entirely in real
  arithmetic are described.
\item Stability and limiting accuracy: Newton’s method and several variants
  follow, with a stability analysis revealing that the variants do not
  suffer the instability that vitiates the Newton iteration. 
\item Scaling the Newton iteration.
\item Numerical Experiments.
\item Iteration via matrix sign function: A class of coupled iterations
  obtained via iterations for the matrix sign function are derived and
  their stability proved. 
\item Special matrices: Linearly convergent iterations for matrices that
  are “almost diagonal”, as well as for M-matrices, are analyzed, and a
  preferred iteration for Hermitian posi- tive definite matrices is given. 
\item Computing small-normed square roots: The issue of choosing from among
  the many square roots of a given matrix is addressed by considering how
  to compute a small-normed square root. 
\item Comparison of methods.
\item Involutory matrices.
\end{enumerate}


\section{Sensitivity and Conditioning}
\subsection{Condition number}
\begin{theorem} [Theorem~3.5 in \cite{high08_FM}]
  \label{thm:fre-deri-of-inv}
  Let $f$ and $f\inv$ both exist and be continuous in an open neighbourhood
  of $X$ and $f(X)$. Assume $L_f$ exists at the neighbourhood and
  nonsingular at $X$. Then, $L_{f\inv}$ exists at $f(X)$, and
  \begin{equation}\notag
    L_f\left(X,L_{f\inv}(f(X),E)\right) = E.
  \end{equation}
\end{theorem}

\begin{example}
  Suppose $f(X) = X^2$, recall that $L_{x^2}(X,E) = XE + EX$. Obviously,
  $f^{-1}(X) = \sqrt{X}$, and let $A = f(X) = X^2$. Then, by
  \ref{thm:fre-deri-of-inv}, we have
  \begin{equation}\notag
    L_{x^2}(X,L_{x^{1/2}}(A,E)) = E,\quad\implies \quad 
    L_{x^{1/2}}(A,E) X + X L_{x^{1/2}}(A,E) = E.
  \end{equation}
  Formally, suppose $g(A) = \sqrt{A}$, then we have $L_g(A,E)$ defined by
  the following matrix equation \comment{Sylvester equation?}
  \begin{equation}\notag
    L_g(A,E)A^{1/2} + A^{1/2}L_g(A,E) = E.
  \end{equation}
\end{example}

Using $\mathrm{vec}(AXB) = (B\tp \otimes A)\mathrm{vec}(X)$, and define
$L_g(A,E) \equiv L$, we have
\begin{equation}\notag
  (X\tp \otimes I + I \otimes X) \mathrm{vec}(L) = \mathrm{vec}(E),
\end{equation}
and then we can deduce that
\begin{equation}\notag
  \fnorm{L} = \tnorm{(X\tp \otimes I + I \otimes X)\inv}.
\end{equation} 
Hence, we have the Frobenius norm (relative) condition number of the matrix
square root at $A$ is
\begin{equation}\notag
  \kappa_{\mathrm{sqrt}}(X) = \frac{\tnorm{(X\tp \otimes I + I \otimes
      X)\inv} \fnorm{A}}{\fnorm{X}}.
\end{equation}
If follows that
\begin{equation}\label{eq:kappa-sqrt}
  \kappa_{\mathrm{sqrt}}(X) \geq \frac{1}{\min_{i,j = 1:n}\abs{\mu_i + \mu_j}}\frac{\fnorm{A}}{\fnorm{X}}
\end{equation}
where $\mu_j$ are the eigenvalues of $X = \sqrt{A}$.

This inequality reveals the situation that the $\kappa_{\mathrm{sqrt}}$
must be large:
\begin{enumerate}
\item When $A$ (hence $X$) has an eigenvalue of small modulus.
\item When the square root is the principal square root and a real $A$ has
  a pair of complex conjugate eigenvalues close to the negative real axis.
  Suppose $\lambda = r\eu^{\im(\pi - \epsilon)}\ (0<\epsilon\ll 1)$ and
  $\wb \lambda = r\eu^{\im(\epsilon - \pi)}$. Then
  \begin{equation}\label{eq:6.3}
    |\lambda^{1/2} + \wb\lambda^{1/2}| =
    r^{1/2}|\eu^{\im(\pi-\epsilon)/2} - \eu^{-\im(\pi-\epsilon)/2}| =
    r^{1/2}O(\epsilon). 
  \end{equation}
\end{enumerate}

If $A$ is normal, then $X$ is normal, by using the fact that a matrix is
normal if and only if it has a spectral decomposition. Then, we have the
equality in \eqref{eq:6.3}.

The formula for $\kappa_{\mathrm{sqrt}}$ allows us to identify the best
conditioned square root of a Hermitian positive definite matrix. Define
$\kappa(A) = \norm{A}\norm{A\inv}$.

\begin{lemma}
  If $A\in\C\nn$ is Hermitian positive definite, and $X$ is any primary
  square root of $A$, then
  \begin{equation}\notag
    \kappa_{\mathrm{sqrt}}(A^{1/2}) =
    \frac{\tnorm{A\inv}^{1/2}}{2}\frac{\fnorm{A}}{\fnorm{A^{1/2}}} \leq
    \kappa_{\mathrm{sqrt}}(X). 
  \end{equation}
  Moreover,
  \begin{equation}\notag
    \frac{1}{2n^{3/2}}\kappa_F(A^{1/2}) \leq
    \kappa_{\mathrm{sqrt}}(A^{1/2}) \leq \frac{1}{2}\kappa_F(A^{1/2}). 
  \end{equation}
\end{lemma}

\begin{proof}
  Since $A$ is positive definite matrix, therefore we can assume that the
  eigenvalues of $A$ satisfies:
  $0 < \lambda_n \le \lambda_{n-1} \le \cdots \le \lambda_2 \le \lambda_1$.
  Also, since $A$ is positive definite, hence it is normal, and
  correspondingly $X$ is normal as well. Therefore we can use the
  equality~(\ref{eq:kappa-sqrt}),
  \begin{equation}\notag
    \kappa_{\mathrm{sqrt}} =
    \frac{1}{2\sqrt{\lambda_n}}\frac{\fnorm{A}}{\fnorm{X}} =
    \frac{\tnorm{A\inv}^{1/2}}{2} \frac{\fnorm{A}}{\fnorm{A^{1/2}}}.
  \end{equation}
  Any other primary square root $X$ has eigenvalues $\mu_j$ with moduli
  $\sqrt{\lambda_j}$, so the upper bound of $\kappa_{\mathrm{sqrt}}(X)$
  follows from
  \begin{equation}\notag
    \min_{i,j = 1:n}\abs{\mu_i + \mu_j} \le \min_{i,j = 1:n}\abs{\mu_i} +
    \abs{\mu_j} \le 2 \sqrt{\lambda_n}
  \end{equation}
  together with the fact that $\fnorm{X}^2 = \sum_{i=1}^n\lambda_i$ is the
  same for all primary square roots of $A$. The upper bound and the lower
  bound of $\kappa_{\mathrm{sqrt}}(A^{1/2})$ are comes from standard norm
  inequalities. \ycite[2002, Chap.~6]{high_ASNA2}.
\end{proof}

The next results shows an elegant bound for the difference between the
principal square roots of two matrices.

\begin{theorem}
  If $A,B\in\C\nn$ are Hermitian positive definite, then for any unitarily
  invariant norm
  \begin{equation}\notag
    \norm{A^{1/2} + B^{1/2}} \leq \frac{1}{\lambda_{\min}(A)^{1/2} +
      \lambda_{\min}(B)^{1/2}} \norm{A - B},
  \end{equation}
  where $\lambda_{\min}$ denotes the smallest eigenvalue.
\end{theorem}

\begin{proof}
  This is a special case of \ycite[1980, Prop.~3.2]{anva80}.
\end{proof}

\subsection{Perturbation}
Suppose $\wt X = X + E$ be an approximation to a square root $X$ of
$A\in\C\nn$, where $\norm{E} \le \epsilon\norm{X}$. Then $\wt X^2 = A + XE
+ EX + E^2$, and this leads to the relative residual bound:
\begin{equation}\notag
  \frac{\norm{A - \wt X^2}}{\norm{A}} \leq (2\epsilon +
  \epsilon^2)\alpha(X), 
\end{equation}
where
\begin{equation}\label{eq:rel-residual-err-bound}
  \alpha(X) = \frac{\norm{X}^2}{\norm{A}} = \frac{\norm{X}^2}{\norm{X^2}}
  \ge 1. 
\end{equation}
The quantity $\alpha(X)$ can be regarded as a condition number for the
relative residual of $X$. If it is large, then \emph{a small perturbation
  of $X$ (such as $fl(X)$, which is the rounded square root) can have a
  relative residual much larger than the size of the relative perturbation.} 
Therefore the conclusion is that, we cannot expect a numerical method to do
better than provide a computed square root $\wh X$ with relative residual
of order $\alpha(\wh X) u$, where $u$ is the unit roundoff.

It is easy to show that 
\begin{equation}\notag
  \frac{\kappa(X)}{\kappa(A)} \leq \alpha(X) \leq \kappa(X).
\end{equation}
\begin{proof}
  Notice that $\norm{X\inv} = \norm{X\inv X\inv X} \le
  \norm{A\inv}\norm{X}$. Hence, we have 
\begin{equation}\notag
  \frac{\norm{X\inv}}{\norm{A\inv}} \le \norm{X} \quad \implies 
\quad \frac{\norm{X}\norm{X\inv}}{\norm{A}\norm{A\inv}} \le
\frac{\norm{X}^2}{\norm{A}},
\end{equation}
this is equivalent as $\kappa(X)/\kappa(A)\le \alpha(X)$.

The right-hand side can be viewed by the following inequality:
\begin{equation}\notag
  \alpha(X) = \frac{\norm{X}^2\norm{X\inv}}{\norm{A}\norm{X\inv}} =
  \frac{\kappa(X)\norm{X}}{\norm{A}\norm{X\inv}} \le
  \frac{\kappa(X)\norm{X}}{\norm{X}} = \kappa(X),
\end{equation}
where the inequality comes from $\norm{X} = \norm{AX\inv} \le
\norm{A}\norm{X\inv}$. Thus a large value of $\alpha(X)$ implies that $X$
is ill-conditioned, and if $A$ is well-conditioned, then $\alpha(X) \approx
\kappa(X)$. If $X$ is normal, then $\alpha(X) = 1$ in 2-norm.
\end{proof}


\section{Schur Method}
\begin{theorem} [Schur decomposition]
  Let $A \in \C^{n \times n}$. Then there exists a unitary matrix $U$ and
  an upper triangular matrix $T$ such that $U^* A U=T$, that is,
  $A=U T U^*$.
\end{theorem}

\begin{theorem}
  [real Schur decomposition]
  Let $A \in \mathbb{R}^{n \times n}$. Then there exists an orthogonal
  matrix $U$ and an upper quasi-triangular matrix $T$ such that
  $U^T A U=T$. Here,
  \begin{equation*}
    T=\left[
      \begin{array}{cccc}
        T_{11} & T_{12} & \cdots & T_{1 m} \\
        0 & T_{22} & \cdots & T_{2 m} \\
        \vdots & \vdots & \ddots & \vdots \\
        0 & 0 & \cdots & T_{m m}
      \end{array}\right],
  \end{equation*}
  where each $T_{i i}$ is either $1 \times 1$ or $2 \times 2$ complex
  conjugate eigenvalues.
\end{theorem}


\subsection{General Schur Decomposition}
Let $A\in\C\nn$ be nonsingular, and let $f(A)$ denotes any primary square
root of $A$. Given a Schur decomposition $A = QTQ\ctp$, where $Q$ is
unitary and $T$ is upper triangular, and $f(A) = Qf(T)Q\ctp$. Hence,
computing the square root of $A$ reduces to computing the square roots
$U = f(T)$ of upper triangular $T$. The $(i,i)$ and $(i,j)(j > i)$ elements
of the equation $U^2 = T$ can be written as
\begin{equation}\label{eq:schur-method}
  \begin{aligned}
    u_{ii}^2 & = t_{ii}, \\
    (u_{ii} + u_{jj})u_{ij} & = t_{ij} - \sum_{k = i + 1}^{j-1}u_{ik}u_{kj}.
  \end{aligned}
\end{equation}
We can compute the diagonal of $U$ and then solve for the $u_{ij}$ either a
superdiagonal at a time or a column at a time. We have the
algorithm~\ref{alg:schur-method}.

\begin{algorithm}
  \caption{(Schur Method). Given a nonsingular $A\in\C\nn$, this algorithm
    computes $X = \sqrt{A}$ via a Schur decomposition, where $\sqrt{\cdot}$
    denotes any primary square root.}
  \label{alg:schur-method}
  \begin{algorithmic}[1]
    \State Compute a (complex) Schur decomposition $A = QTQ\ctp$. \State
    $u_{ii} = \sqrt{t_{ii}},\ i = 1:n$ \For {$j = 2:n$} \For
    {$i = j-1:-1:1$} \State
    $\displaystyle u_{ij} = \frac{t_{ij} -
      \sum_{k=i+1}^{j-1}u_{ik}u_{kj}}{u_{ii} + u_{jj}}$ \EndFor \EndFor
    \State $X = QUQ\ctp$
  \end{algorithmic}
\end{algorithm}

The \emph{cost}: $25n^3$ flops for the Schur decomposition plus $n^3/3$ for
$U$ and $3n^3$ to form $X$, which gives $28 \frac{1}{3}n^3$ flops in total.

Algorithm~\ref{alg:schur-method} generates all the primary square roots of
$A$ as different choices of sign in
$u_{ii} = \sqrt{t_{ii}} = \pm t_{ii}^{1/2}$ are used.

\subsection{Real Schur Decomposition}
If $A$ is real but has some nonreal eigenvalues, then
Algorithm~\ref{alg:schur-method} uses complex arithmetic. This is
\emph{undesirable}, because (i) complex arithmetic is more expensive than
real arithmetic, and (ii) rounding errors may cause a computed result to be
produced with nonzero imaginary part. We can use the real Schur
decomposition instead to avoid complex arithmetic.

Let $A \in \R\nn$ have the real Schur decomposition $A = QRQ\tp$, where $Q$
is orthogonal and $R$ is upper quasi-triangular with $1\times 1$ and
$2\times 2$ diagonal blocks. Then $f(A) = Qf(R)Q\tp$, where $U = f(R)$ is
upper quasi-triangular with the same block structure as $R$. The equation
$U^2 = R$ can be written as in an analogous way as (\ref{eq:schur-method}):
\begin{align}
  U_{ii}^2 &= R_{ii} \notag \\ 
  U_{ii}U_{ij} + U_{ij}U_{jj} & = R_{ij} -
                                \sum_{k=i+1}^{j-1}U_{ik}U_{kj}. \label{eq:schur-slyvester} 
\end{align}
Once the diagonal blocks $U_{ii}$ are computed, we can use
\eqref{eq:schur-slyvester} to compute the remaining blocks $U_{ij}$ a block
superdiagonal or a block column at a time. The condition for the Sylvester
equation~(\ref{eq:schur-slyvester}) to have a unique solution $U_{ij}$ is
that \emph{$U_{ii}$ and $-U_{jj}$ have no eigenvalue in common, and this is
  guaranteed for any primary square root when $A$ is
  nonsingular.\comment{why?}} When neither $U_{ii}$ nor $U_{jj}$ is a
scalar, (\ref{eq:schur-slyvester}) can solved by writing in the form
\begin{equation}\notag
  (I \otimes U_{ii} + U_{jj}\tp \otimes I) \mathrm{vec}(U_{ij}) =
  \mathrm{vec}(R_{ij} - \sum_{k=i+1}^{j-1}U_{ik}U_{kj}),
\end{equation}
which is a linear system $Ax = b$ of order 4 that can be solved by Gaussian
elimination with partial pivoting.

We now consider the computation of $\sqrt{R_{ii}}$ for $2\times 2$ blocks
$R_{ii}$, which is necessarily have distinct complex conjugate eigenvalues.
(refer to the properties of the
\href{https://nhigham.com/2022/05/11/what-is-a-schur-decomposition/}{Schur
  decomposition}).

\begin{lemma}\label{lemma:2by2-primary-sqrt}
  Let $A\in\R^{2\times 2}$ have distinct complex conjugate eigenvalues.
  Then $A$ has four square roots, and all of them are primary functions of
  $A$. Two of them are real, with complex conjugate eigenvalues; two are
  pure imaginary, with eigenvalues that are not complex conjugates.
\end{lemma}

\begin{proof}
  Since $A$ has distinct eigenvalues $\theta \pm \im \mu$, then $A$ has
  four square roots, all of them are functions of $A$~\ycite[2008,
  Thm~1.26]{high08_FM}. Remain to construct them. Consider the following
  matrix construction:
  \begin{equation}\notag
    Z\inv A Z = \diag(\lambda, \wb \lambda) = \theta I + \im \mu K, \quad 
    K =
    \begin{bmatrix}
      1 & 0 \\ 0 & -1 
    \end{bmatrix}.
  \end{equation}
  Then $A = \theta I + \mu W$, where $W = \im Z K Z\inv$, and since
  $\theta, \mu \in \R$, it follows that $W \in \R^{2\times 2}$. Suppose
  $(\alpha + \im \beta)^2 = \theta + \im \mu$, then all four square roots
  of $A$ are given by $X = ZDZ\inv$, where
  $D = \pm \diag(\alpha + \im \beta, \pm (\alpha - \im\beta))$. Using
  previous notation, we can categorise the diagonal matrix $D$ into two
  distinct cases:
  \begin{equation}\notag
    D_1 = \pm (\alpha I + \im \beta K),\qquad 
    D_{2} = \pm \im (\beta I - \im \alpha K).
  \end{equation}
  Thus, we can reconstruct the matrix square root $X$ in the following two
  ways: Real matrix with complex conjugate eigenvalues
  $\lambda(X_1) = \pm(\alpha + \im \beta, \alpha - \im \beta)$.
  \begin{equation}\notag
    X_1 = ZD_1Z\inv = \pm (\alpha I + \beta \im ZKZ\inv) = \pm(\alpha I +
    \beta W) \in \R^{2\times 2}.  
  \end{equation}
  Pure imaginary matrix with non complex conjugate eigenvalues
  $\lambda(X_2) = \pm(\alpha + \im \beta , -\alpha + \im \beta)$.
  \begin{equation}\notag
    X_2 = ZD_2Z\inv = \pm \im (\beta I - \alpha \im ZKZ\inv) = \pm \im
    (\beta I - \alpha W).
  \end{equation}
\end{proof}

The proof gives a way to construct $R_{ii}^{1/2}$, writting
\begin{equation}\notag
  R_{ii} = 
  \begin{bmatrix}
    r_{11} & r_{12} \\ r_{21} & r_{22}
  \end{bmatrix}
\end{equation}
the eigenvalues of $R_{ii}$ are $\theta \pm \im \mu$, where
\begin{equation}\notag
  \theta = \frac{1}{2}(r_{11} + r_{22}),\qquad \mu = \frac{1}{2}\left(
    -(r_{11} - r_{22})^2 - 4r_{21}r_{12} \right)^{1/2}.
\end{equation}
The former equation is constructed using $\tr(A) = \sum_i \lambda_i(A)$.
The latter equation is constructed using
$(\theta + \im \mu)(\theta - \im\mu) = \theta^2 + \mu^2$.

We now require $\alpha$ and $\beta$ such that
$(\alpha + \im\beta)^2 = \theta + \im\mu$. A stable way to compute them is
as follows:

\begin{algorithm}
  \caption{This algorithm computes the square root $\alpha + \im\beta$ of
    $\theta + \im\mu$ with $\alpha \geq 0$.}
  \begin{algorithmic}[1]
    \State If $\theta = 0$ and $\mu = 0$, then $\alpha = \beta = 0$, quit,
    end. \State
    $\displaystyle t = \left( \frac{\abs{\theta} + (\theta^2 +
        \mu^2)^{1/2}}{2} \right)^{1/2}$ \State if $\theta \ge 0$ \State
    \quad $\alpha = t, \beta = \mu/(2\alpha)$ \State else \State \quad
    $\beta = t, \alpha = \mu/(2\beta)$ \State end
  \end{algorithmic}
\end{algorithm}

Finally, the real square roots of $R_{ii}$ are obtained from
\begin{align}
  U_{ii} & = \pm (\alpha I + \frac{1}{2\alpha}(R_{ii} - \theta I)) \notag \\ 
         & = \pm 
           \begin{bmatrix}
             \alpha + \frac{1}{4\alpha}(r_{11} - r_{22}) & \frac{1}{2\alpha}r_{12}
             \\
             \frac{1}{2\alpha} r_{21} & \alpha - \frac{1}{4\alpha}(r_{11} - r_{22})
           \end{bmatrix}. \label{eq:6.9}
\end{align}

Before present the algorithm, let us discuss the number of real square
roots for arbitrary $A\in\R\nn$.

\begin{theorem}
  Let $A\in\R\nn$ be nonsingular. If $A$ has a real negative eigenvalue,
  then $A$ has no real square root that is primary function of $A$.

  If $A$ has no real negative eigenvalues, then there are precisely
  $2^{r+c}$ \emph{real} primary square roots of $A$, where $r$ is then
  number of distinct real eigenvalues and $c$ is the number of distinct
  complex conjugate eigenvalue pairs.
\end{theorem}

\begin{proof}
  Let $A$ has a real Schur decomposition. Since $f(A) = Qf(R)Q\tp$, $f(A)$
  is real if and only if $f(R)$ is real. If $A$ has a real negative
  eigenvalue, $R_i = (r_{ii})$ say, then $f(R_i)$ is necessarily nonreal,
  and this gives the first part of the proof. 

  If $A$ has no real negative eigenvalues, consider the $2^s$ primary
  square roots  of $A$ described in Theorem~1.26. We have $s = r + 2c$,
  i.e. $A$ has $2^r$ real primary square roots from real, distinct
  eigenvalues; also, a half of $2^{2c}$ primary square roots from complex
  conjugate eigenvalues are \emph{real} according to
  Lemma~\ref{lemma:2by2-primary-sqrt}, hence precisely $2^{r + c}$ of its
  primary square roots are real. 
\end{proof}

\begin{algorithm}
  \caption{(real Schur method). Given $A \in \R\nn$ with no eigenvalues on
    $\R^{-}$, this algorithm computes $X = \sqrt{A}$ via a Schur
    decomposition, where $\sqrt{\cdot}$ denotes any real primary square
    root.}
  \label{alg:real-schur-method}
  \begin{algorithmic}
    \State Compute a real Schur decomposition, $A = QRQ\tp$, where $R$ is
    block $m\times m$.
    \State Compute $U_{ii} = \sqrt{R_{ii}}$, $i = 1:m$, using
    (\ref{eq:6.9}) whenever $R_{ii}$ is $2\times 2$.
    \For {$j = 1:m$}
        \For{$i = j -1 : -1 : 1$}
            \State Solve $U_{ii}U_{ij} + U_{ij}U_{jj} = R_{ij} - \sum_{k =
              i + 1}^{j - 1}U_{ik}U_{kj}$ for $U_{ij}$.
        \EndFor
    \EndFor
    \State{$X = QUQ\tp.$}
  \end{algorithmic}
\end{algorithm}

\begin{remark}
  \ 
  
  \begin{enumerate}
  \item The principal square root is computed if the principal square root
    is taken at line 2, which for $2\times 2$ blocks means taking the
    positive sign in (\ref{eq:6.9}).
  \item Second, as for \ref{alg:schur-method}, it is necessary that
    whenever $R_{ii}$ and $R_{jj}$ have the same eigenvalues, we take the
    same square root. 
  \end{enumerate}
\end{remark}

\subsection{Numerical stability}
Now we consider the numerical stability of Algorithm~\ref{alg:schur-method}
and \ref{alg:real-schur-method}. A straightforward rounding error analysis
shows that the computed square root $\wh U$  of $T$ in
Algorithm~\ref{alg:schur-method} satisfies 
\begin{equation}\notag
  \wh U^2 = T + \varDelta T, \quad \abs{\varDelta T} \leq \wt \gamma_n
  \abs{\wh U}^2 
\end{equation}
where the inequality is to be interpreted elementwise.  Computation of the
Schur decomposition by QR algorithm is a backward stable
process~\ycite[2013, Sec.~7.5.6]{golo13_MC} and standard error analysis
leads to the overall result
\begin{equation}\notag
  \wh X^2 = A + \varDelta A,\quad \fnorm{\varDelta A} \leq \wt
  \gamma_{n^3}\fnorm{\wh X}^2
\end{equation}
which can be expressed as 
\begin{equation}\label{eq:6.10}
  \frac{\fnorm{A - \wh X^2}}{\fnorm{A}} \leq \wt \gamma_{n^3}\alpha_F(\wh X)
\end{equation}
where $\alpha$ is defined in (\ref{eq:rel-residual-err-bound}). The same
conclusion holds for Algorithm~\ref{alg:real-schur-method}, which can be
shown to satisfy the same error bound~(\ref{eq:6.10}).

\section{Newton's Method and Its Variants}

\subsection{Newton and Commutativity}

Newton method for solving $X^2 = A$ can be derived as follows. Let $Y$ be
an approximate solution, and set $Y + E = X$, where $E$ is to be
determined. Then $A = (Y + E)^2 = Y^2 + EY + YE + E^2$. Dropping the second
order term in $E$ leads to the Newton's method:
\begin{equation}
  \label{eq:newton-method-sqrt}
  \left.
  \begin{aligned}
     X_0 \text{ given} & \\
     \text{Solve } X_kE_k + E_kX_k & = A - X_k^2 \\
     X_{k+1} & = X_{k} + E_k     
  \end{aligned} \right\} \quad k = 0,1,2,\dots,
\end{equation}

At each iteration, a Sylvester equation mush be solved for $E_k$. The
standard way of solving the Sylvester equation is via Schur decomposition
of the coefficient matrices, which in this case are both $X_k$. But the
Schur method of the previous section can compute a square root with just
one Schur decomposition, so Newton's method is unduly expensive in the form
\eqref{eq:newton-method-sqrt}. 

The following lemma enable us to reduce the cost. Note that the $E_k$ in
\eqref{eq:newton-method-sqrt} is well defined, that is, the Sylvester
equation is nonsingular, if and only if $X_k$ and $-X_{k}$ have no
eigenvalues in common. 

\begin{lemma}
  Suppose that in the Newton iteration \eqref{eq:newton-method-sqrt}, $X_0$
  commutes with $A$ and all the iterates are well-defined. Then for all
  $k$, $X_k$ commutes with $A$ and $X_{k+1} = \frac{1}{2}(X_k + X_k\inv A)$.
\end{lemma}

\begin{proof}
  We prove this by induction. Firstly, we notice that if $X_k$ and $A$ are
  commute, then  there is a trivial solution for the Sylvester equation
  $X_kE_k + E_k X_k = A - X_k^2$, which is $E_k = \frac{1}{2}(X_k\inv A -
  X_k)$ and this solution is clearly commute with $A$. Hence, we will stick
  to this solution and start for induction.

  \noindent\emph{Induction Statement.} For $k = 1,2,\dots,$ $X_k$ commute
  with $A$ and $X_{k + 1} = \frac{1}{2}(X_k + X_k\inv A)$.

  \noindent\emph{Base case.} For $k = 1$, we have $X_1 = X_0 + E_0$, where
  $E_0 = \frac{1}{2}(X_0^{-1} A - X_0)$, and  
  \begin{equation}\notag
    X_1 = X_0 + \frac{1}{2}(X_0^{-1} A - X_0) = \frac{1}{2}(X_0\inv A + X_0).
  \end{equation}
  Here, $X_1$ is commute with $A$, and $E_0$ is commute with $A$. 

  \noindent\emph{Inductive step.} Suppose the statement is true for $k =
  n-1$, i.e. $X_{n-1}$ commute with $A$ and $X_n = \frac{1}{2}( X_{n-1}
  \inv A + X_{n-1})$ which is also commute with $A$ by $AX_{n -1} =
  X_{n-1}A$.  By solution of the  Sylvester equation, we have $E_n =
  \frac{1}{2}(X_n\inv A - X_n)$, and this is clearly commute with $A$ by
  the commutativity between $X_n$ and $A$. Moreover, $X_{n + 1} =
  \frac{1}{2}(X_n\inv A + X_{n})$ which is also commute with $A$ by $A$ and
  $X_n$ are commute. The proof is then complete.
\end{proof}

This lemma shows that if $X_0$ is chosen to commute with $A$, then all the
$X_k$ and $E_{k}$ in \eqref{eq:newton-method-sqrt} commute with $A$,
permitting a good simplification of the iteration. The most common choice
of $X_0$ is $A$, giving the Newton iteration 
\begin{mybox}{}
  \begin{center}
    \textsf{Newton iteration (matrix square root)}
  \end{center}
  \begin{equation}\label{eq:6.12}
    X_{k + 1} = \frac{1}{2}(X_k + X_k\inv A), \qquad X_0 = A.
  \end{equation}
\end{mybox}

\subsection{Link to the Matrix Sign Function}

If $A$ is nonsingular, standard convergence theory for Newton's method
allows us to deduce quadratic convergence of \eqref{eq:newton-method-sqrt}
to a primary square root for $X_0$ sufficiently close to square root, since
the Fr\'{e}chet derivative of $F(X) = X^2 - A$ is nonsingular at a primary
square root. The next result shows the unconditional quadratic convergence
of \eqref{eq:6.12} to the \emph{principal} square root. Moreover, it shows
that \eqref{eq:6.12} is equivalent to the Newton sign iteration
\eqref{eq:nt-sign-iter} 

\begin{mybox}{}
  \begin{center}
    \textsf{Newton iteration (matrix sign function)}
  \end{center}
  \begin{equation}
    \label{eq:nt-sign-iter}\tag{Sign}
    X_{k + 1} = \frac{1}{2}(X_k + X_k\inv), \quad X_0 = A.
  \end{equation}
\end{mybox}

\begin{theorem}[Convergence of Newton square root iteration]
  \label{thm:6.9}
  Let $A\in\C\nn$ has no eigenvalues on $\R^-$. The Newton square root
  iterates $X_k$ from \eqref{eq:6.12} with any $X_0$ that commutes with $A$
  are related to the Newton sign iterates 
  \begin{equation}\notag
    S_{k + 1} = \frac{1}{2} (S_k + S_k\inv),\quad S_0 = A^{-1/2}X_0,
  \end{equation}
  by $X_k \equiv A^{1/2}S_k$. Hence provided that $A^{-1/2}X_0$ has no pure
  imaginary eigenvalues, the $X_k$ are defined and $X_k$ converges
  quadratically to $A^{1/2}\sign(A^{-1/2} X_0)$.

  In particular, if the spectrum of $A^{-1/2}X_0$ lies in the right
  half-plane then $X_k$ converges quadratically to $A^{1/2}$ and, for any
  consistent norm, 
  \begin{equation}\label{eq:6.13}
    \norm{X_{k+1} - A^{1/2}} \leq \frac{1}{2} \norm{X_k\inv} \norm{X_k -
      A^{1/2}}^2 
  \end{equation}
\end{theorem}

\begin{proof}
  We first note that any matrix that commutes with $A$ commutes with
  $A^{\pm 1/2}$ since it's a polynomial of $A$. We have $X_0 = A^{1/2}S_0$,
  hence $S_0$ commute with $A$.

  Assume that $X_k = A^{1/2}S_k$, and $S_k$ commute with $A$, then $S_k$
  commute with $A^{1/2}$, and
  \begin{equation}
    \notag
    X_{k + 1} = \frac{1}{2}(A^{1/2}S_k + S_k\inv A^{-1/2}A) = A^{1/2} \cdot
    \frac{1}{2}(S_k + S_k \inv) = A^{1/2}S_{k+1},
  \end{equation}
  and $S_{k + 1}$ clearly commute with $A$. Hence $X_k \equiv A^{1/2}S_k$
  by induction. Then using \ycite[2008, Theorem~5.6]{high08_FM},
  \begin{equation}
    \notag
    \lim_{k \to \infty}X_k = A^{1/2}\lim_{k \to \infty}S_k =
    A^{1/2}\sign(S_0) = A^{1/2}\sign(A^{-1/2}X_{0}), 
  \end{equation}
  and the quadratic convergence of $X_k$ follows from that of $S_k$.

  For the last part, if $S_0 = A^{-1/2}X_0$ has spectrum in the right
  half-plane then $\sign(S_0) = I$ and hence $X_k \to A$. Using the
  commutativity of the iterates with $A$, it is easy to show that 
  \begin{equation}\label{eq:6.14}
    X_{k + 1} \pm A^{1/2} = \frac{1}{2}X_k\inv (X_k \pm A^{1/2})^2
  \end{equation}
  which, with minus sign, gives \eqref{eq:6.13}.
\end{proof}


\begin{remark} \  
  \begin{enumerate}
  \item An implication of Theorem~\ref{thm:6.9} of theoretical interest,
    which can also be deduced from the connection with the full Newton
    method, is that \eqref{eq:6.12} converges to $A^{1/2}$ for any $X_0$
    that commutes with $A$ and is sufficiently close to $A^{1/2}$.
  \item 
  It is worth noting that the sequence $X_k$ from \eqref{eq:6.12} may be
  well-defined when that for the full Newton method
  \eqref{eq:newton-method-sqrt} is not. No analogue of the condition in
  \ref{thm:6.9} guaranteeing that the $X_k$ is well-defined is available
  for \eqref{eq:newton-method-sqrt}. 
  \item 
  This analysis using the Newton sign iteration is more powerful than the
  analysis using the Jordan canonical form in
  \cite[Section~4.9.3]{high08_FM}. If $X_0$ is only known to commute with
  $A$, then $X_k$ do not necessarily share the same Jordan block as $A$, so
  the analysis cannot break down to one single Jordan form.
\item 
  Note that, from \cite[Section~5.3]{high08_FM}, the Newton iteration for
  $\sign(A)$ requires more iterations if $A$ has eigenvalues close to the
  imaginary axis. Theorem~\ref{thm:6.9} therefore implies that the Newton
  iteration \eqref{eq:6.12} requires more iterations if $A$ has eigenvalues
  close to the negative real axis. 
\item 
  When $A$ is positive definite, the convergence of \eqref{eq:6.12} is
  monotonic from above in the positive semidefinite ordering.
\item 
  It's interesting to consider how \eqref{eq:6.12} behaves when $X_0$ does
  not commute with $A$, although commutativity is assumed in the
  derivation. Lack of commutativity can cause quadratic convergence, and
  even convergence itself, to be lost. 
  \end{enumerate}
\end{remark}


\subsection{Other Variants of Newton Iteration}
A coupled version of \eqref{eq:6.12} can be obtained by defining
$Y_k = A\inv X_k$. Then $X_{k + 1} = \frac{1}{2}(X_k + Y_k\inv)$ and
$Y_{k+1} = A\inv X_{k+1} = \frac{1}{2}(Y_k + Y_k\inv)$ on using the fact
that $X_k$ commute with $A$. This is the iteration of Denman and
Beavers~\cite{debe76}
\begin{mybox}{}
  \begin{center}
    \textsf{DB iteration}
  \end{center}
  \begin{equation}
    \label{eq:6.15}
    \begin{aligned}
      X_{k + 1} & = \frac{1}{2}(X_k + Y_k\inv), \quad X_0 = A, \\
      Y_{k + 1} & = \frac{1}{2}(Y_k + X_k\inv), \quad Y_0 = I.
    \end{aligned}
  \end{equation}
\end{mybox}
Under the condition of \ref{thm:6.9}
\begin{equation}\label{eq:6.16}
  \lim_{k \to \infty} X_k = A^{1/2}, \qquad \lim_{k\to \infty}Y_k =
  A^{-1/2}. 
\end{equation}

Defining $M_k = X_kY_k$, we have
\begin{equation}\notag
  M_{k+1} = \frac{1}{2}(X_kY_k\inv)\frac{1}{2}(Y_k + X_k\inv) =
  \frac{1}{4}(2I + M_k + M_k\inv),
\end{equation}
gives the product form of the DB iteration, identified by Cheng, Higham,
Kenney and Laub~\cite{chkl01} in which we iterates with $M_k$ and either
$X_k$ or $Y_k$:
\begin{mybox}{}
  \begin{center}
    \textsf{Product form DB iteration}
  \end{center}
  \begin{equation}
    \label{eq:6.17}
    \begin{aligned}
      M_{k + 1} &= \frac{1}{2}\left(I + \frac{M_k + M_k\inv}{2}\right),
                  \quad M_0 = A,\\ 
      X_{k + 1} &= \frac{1}{2}X_k(I + M_k\inv), \quad X_0 = A, \\ 
      Y_{k + 1} &= \frac{1}{2}Y_k(I + M_k\inv), \quad Y_0 = I.
    \end{aligned}
  \end{equation}
\end{mybox}

Clearly, $\lim_{k\to \infty} M_k = I$. The product form DB iteration has
the advantage in efficiency over DB iteration that it has trade one of the
matrix inversions for a matrix multiplication.

Another attraction of \eqref{eq:6.17} is that, a convergence test can be
based on the error $\norm{M_k-I}$, which is available free of charge.

Yet another variant of \eqref{eq:6.12} can be derived by noting that
\begin{equation}\notag
  \begin{aligned}
    E_{k + 1} & = \frac{1}{2}(X_{k + 1}\inv A - X_{k + 1}) \\ 
              & = \frac{1}{2} (X_{k + 1}\inv)(A- X_{k + 1}^2) \\
              & = \frac{1}{2} (X_{k + 1}\inv)\left(A - \frac{1}{4}(X_k + X_k\inv
                A)\right) \\
              & = \frac{1}{2} X_{k + 1}\inv \left( \frac{2A - X_k^2 - X_k^{-2}A^2}{4}
                \right) \\ 
              & = -\frac{1}{2} X_{k + 1}\inv \frac{(X_k - X_k\inv
                A)^2}{4}\\ 
              & = -\frac{1}{2}X_{k + 1}\inv E_k^2 = -\frac{1}{2}E_kX_{k +
                1}\inv E_k. 
  \end{aligned}
\end{equation}

Setting $Y = 2E_k$ and $Z_k = 4X_{k + 1}$, we obtain the iteration
\begin{mybox}{}
  \begin{center}
    \textsf{CR iteration}
  \end{center}
  \begin{equation}
    \label{eq:6.19}
    \begin{aligned}
      Y_{k + 1}& = - Y_k Z_k\inv Y_k,\quad Y_0 = I - A,\\ 
      Z_{k + 1}& = Z_k + 2Y_{k + 1},\quad Z_0 = 2(I + A)
    \end{aligned}
  \end{equation}
\end{mybox}

From the derivation, we have $Y_k \to 0$ and $Z \to 4A^{1/2}$. This
iteration is derived in a different way by Meini~\cite{mein04}.

A minor variation of \eqref{eq:6.19} is: set $X_k = Z_k/4$ and
$E_k = Y_{k + 1} /2$, then \eqref{eq:6.19} becomes

\begin{mybox}{}
  \begin{center}
    \textsf{IN iteration}
  \end{center}
  \begin{equation}
    \label{eq:6.20}
    \begin{aligned}
      X_{k +1 }& = X_k + E_k,\quad X_0 = A, \\
      E_{k + 1} & = -\frac{1}{2}E_k X_{k + 1}\inv E_k, \quad E_0 = \frac{1}{2}(I - A).
    \end{aligned}
  \end{equation}
\end{mybox}

Here $X_{k} \to A^{1/2}$ and $E_k \to 0$. This incremental form of the
Newton iteration, suggested by Iannazzo, is of interest because it update
$X_k$ by a correction that is small and accurately computable. 

The computational cost of the Newton iteration and its variants is compared
in Table~\ref{tab:6.1}.

\begin{table}[H]
  \centering
  \begin{tabular}{r|cc}
    \toprule 
    Iteration & Operations & Flops \\ \midrule
    Newton, \eqref{eq:6.12} & $D$ & $8n^3/3$ \\
    DB, \eqref{eq:6.15} & $2I$ & $4n^3$ \\
    Product DB, \eqref{eq:6.17} & $M + I$ & $4n^3$ \\
    CR, \eqref{eq:6.19} & $M + D$ & $14n^3/3$ \\
    IN, \eqref{eq:6.20} & $M + D$ & $14n^3/3$\\ \bottomrule
  \end{tabular}
  \caption{Cost per iteration of matrix square root iterations}
  \label{tab:6.1}
\end{table}

Here $M$ denotes a matrix multiplication, $I$ denotes a matrix inversion,
and $D$ denotes a solution of a multiple right-hand side linear system.
Clearly, the Newton iteration~\eqref{eq:6.12} is the least expensive
iteration. 


%%%%%%%%%%%%%%%%%%%%%%%%%%% Bibliographies %%%%%%%%%%%%%%%%%%%%%%%%%%%

\newpage
\bibliographystyle{/Users/clement/Dropbox/bibtex/nj-plain}
\bibliography{/Users/clement/Dropbox/bibtex/ref.bib}
%% APPENDICES

\end{document}

%%% Local Variables:
%%% mode: latex
%%% TeX-master: t
%%% End:
