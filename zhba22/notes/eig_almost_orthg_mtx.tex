\documentclass{article}

\def\ntitle {Reading Notes \cite{ZB2022AMPJMftSEP}}
% \def\nauthor{ } % default author is Zhengbo Zhou
% \def\notes{ } % default is the submitted version
% \def\ndate{ } % default is today's date
\input{clem.tex}
\DeclareMathOperator{\off}{off}
\newcommand{\Egap}[1]{\min_{\lambda_i({#1}) \neq \lambda_j({#1})} \abs{\lambda_i({#1}) - \lambda_j({#1})}}

%%%%%%%%%%%%%%%%%%%%%%%%%%%%%%%%%%%%%%%%%%%%%%%%%%%%%%%%%%%%%%%%%%%%%%%%
\newtcbtheorem[number within=section]%
{claim} % \begin..
{Claim} % Title
{
    colback=white,
    colframe=black,
    boxrule=0.75pt,
    breakable,
    arc=0pt,
    outer arc=0pt,
} % Style - default
{claim} % label prefix; cite as ``theo:yourlabel''

%%%%%%%%%%%%%%%%%%%%%%%%%%%%%%%%%%%%%%%%%%%%%%%%%%%%%%%%%%%%%%%%%%%%%%%%


\begin{document}
\maketitle
% \tableofcontents
\thispagestyle{firstpage} 

\begin{abstract}
    % 1. What did the authors try to accomplish?
    The authors try to implement a Jacobi algorithm which utilizes the
    benefit of fastness of low precision. It first computes the
    approximate eigendecomposition in low precision, then orthogonalize
    it using MGS approach and apply to original matrix as a
    preconditioner, and finally compute the eigensystem of
    preconditioned matrix. 
    
    The paper also provides the bound on distance
    of low precision eigenmatrix and its orthogonal QR factor, a bound
    on $\mathrm{off}(\text{Preconditioned Matrix})$, and a sufficient
    condition for the Jacobi algorithm to have quadratic convergence
    that related to $\operatorname{Egap}(A)$ and
    $\operatorname{Egap}(Q\tp AQ)$. 
    % 2. What were the key elements of the approach
    The key element is the preconditioning method. Notice that the
    quadratic convergence is reaily presented by~\cite{Kempen1966}, and
    by preconditioning, the preconditioned matrix is automatically
    satisfies the condition presented in \cite{Kempen1966}.
    % 3. What can you use yourself?
    We can study the way of constructing $\norm{Z-Q}$ and way of
    constructing error bounds.
    % 4. What other references do you want to follow?

\end{abstract}

%%%%%%%%%%%%%%%%%%%%%%%%%%%% MAIN ARTICLE %%%%%%%%%%%%%%%%%%%%%%%%%%%%
\section{Summary and Outline of Paper}
\begin{lemma}
    \label{lemma.1.1}
    Lemma 2.1 Let $A \in \mathbb{R}^{m \times n}$ with
    $\operatorname{rank}(A)=n$. Suppose the $M G S$ method computes the
    approximate $Q R$ factorization $A \approx \widehat{Q} \widehat{R}$ in
    precision $v$, where $\widehat{R} \in \mathbb{R}^{n \times n}$ is upper
    triangular and $\widehat{Q} \in \mathbb{R}^{m \times n}$. Then there
    exist constants $\eta_i \equiv \eta_i(m, n)$ for $i=1,2,3$ such that 
    $$
    \|A-\widehat{Q} \widehat{R}\| \leq \eta_1\|A\| v, \quad\|\widehat{Q}^T \widehat{Q}-I_n\| \leq \eta_2 \kappa(A) v, 
    $$
    and $\widehat{Q}+\delta \widehat{Q}$ is orthogonal with $\|\delta \widehat{Q}\|
    \leq \eta_3 \kappa(A) v$, where $\kappa(A)=\sigma_1(A) /
    \sigma_{\min }(A)$ is the condition number of $A$. 
\end{lemma}

\begin{lemma}\label{lemma.2-2}
    Lemma 2.2 Let $A \in \mathbb{R}^{n \times n}$ be a real symmetric
    matrix. The computed symmetric eigenvalue decomposition (SED) $A
    \approx \widehat{P} \widehat{\Lambda} \widehat{P}^T$ with $\widehat{P} \in
    \mathbb{R}^{n \times n}$ and
    $\widehat{\Lambda}=\operatorname{diag}\left(\widehat{\lambda}_1, \ldots,
    \widehat{\lambda}_n\right) \in \mathbb{R}^{n \times n}$ via any
    eigensolver in LAPACK or EISPACK in precision $v$ is nearly the
    exact symmetric Schur decomposition of $A+E$, i.e.,
    $$
        A+E=(\widehat{P}+\delta \widehat{P}) 
        \widehat{\Lambda}(\widehat{P}+\delta \widehat{P})^T,
    $$
    where $\|E\| \leq p(n)\|A\| v$ and $\widehat{P}+\delta \widehat{P}$ is
    orthogonal with $\|\delta \widehat{P}\| \leq p(n) v$. Here, $p(n)$ is a
    modestly growing function of $n$. 
\end{lemma}

\begin{lemma}
    Lemma 2.3 Let $A \in \mathbb{R}^{m \times n}$ be a real matrix $(m
    \geq n)$. The computed $S V D A \approx \widehat{U} \widehat{\Sigma}
    \widehat{V}^T$ with $\widehat{U} \in \mathbb{R}^{m \times m}, \widehat{V} \in
    \mathbb{R}^{n \times n}$, and
    $\widehat{\Sigma}=\operatorname{diag}\left(\widehat{\sigma}_1, \ldots,
    \widehat{\sigma}_n\right) \in \mathbb{R}^{m \times n}$ via any SVD
    solver in LAPACK, LINPACK or EISPACK in precision $v$ is nearly the
    exact SVD of $A+E$, i.e., 
    $$
        A+E=(\widehat{U}+\delta \widehat{U}) \widehat{\Sigma}(\widehat{V}+\delta \widehat{V})^T
    $$
    where $\|E\| \leq p(m, n)\|A\| v$ and $\widehat{U}+\delta \widehat{U}$ and
    $\widehat{V}+\delta \widehat{V}$ are both orthogonal with $\|\delta
    \widehat{U}\| \leq p(m, n) v$ and $\|\delta \widehat{V}\| \leq p(m, n) v$.
    Here, $p(m, n)$ is a modestly growing function of $m$ and $n$. 
\end{lemma}

\begin{lemma}
    Lemma 2.4 If $A$ and $A+E$ are $n \times n$ real symmetric matrices,
    then 
    $$
     \left|\lambda_k(A+E)-\lambda_k(A)\right| \leq\|E\|,
    $$
    for $k=1, \ldots, n$.
\end{lemma}

\begin{lemma}
    \label{lemma.3-3}
    Lemma 3.3 Let $A^{(k)}$ be the matrix $A$ after $k$ Jacobi updates in Algorithm $3.2$ with $A^{(0)}=A$. Suppose
    $$
    \min _{\lambda_i\left(A^{(0)}\right) \neq \lambda_j\left(A^{(0)}\right)}\left|\lambda_i\left(A^{(0)}\right)-\lambda_j\left(A^{(0)}\right)\right| \geq d>0 .
    $$
    Moreover, if we reach a point that $\|\operatorname{off}\left(A^{(k)}\right)\|_F<d /(4 \sqrt{2})$, then we have
    $$
    \|\operatorname{off}\left(A^{(k+N)}\right)\|_F \leq \frac{\sqrt{34 / 9}}{d}\|\operatorname{off}\left(A^{(k)}\right)\|_F^2
    $$
    where $N=n(n-1) / 2$
\end{lemma}

\section{Outline of Proofs}

\begin{lemma}
    \label{lemma.4-1}
    Suppose that $Z \in \mathbb{R}^{n \times n}$ in Step 1 of Algorithm
    $4.1$ is computed by any eigensolver in LAPACK or EISPACK in
    precision $v$ and $Q \in \mathbb{R}^{n \times n}$ in Step 2 of
    Algorithm $4.1$ is computed by using the MGS method to $Z$ in
    precision $\omega(\omega \ll v)$. Then there exist constants $h_i
    \equiv h_i(n)$ for $i=1,2$ such that
    $$
        \|Z-Q\|_F \leq h_1 v+h_2 \omega
    $$
\end{lemma}
%%%%%%%%%%%%%%%%%%%%%%%%%%%%%%%%%%%%%%%%%%%%%%%%%%%%%%%%%%%%%%%%%%%%%%%%

\begin{proof}
    [Outline of proof]
    \ 

    \begin{itemize}
        \item Rewrite $\|Z - Q\|_F = \|QR+F_1-Q\| \leq
        \|Q\|\|R-I\|_F+\|F_1\|$.
        \item Construct bounds of $\|Q\|$ and $\|F_1\|$ by given theorems.
        \item Rewrite $\|R-I\|_F = \sqrt{\sum_{i=1}^n (r_{ii}-1)^2 + \sum_{i
        < j}r_{ij}^2}$.
        \item Construct bounds for $r_{ii}$ and $r_{jj}$.
        \item Notice $1/\|R\inv\| \leq \abs{r_{ii}} \leq \|R\|$, hence we
        need to construct bounds on $\|R\|$ and $\|R\inv\|$.
        \item Moreover, notice ${\abs{r_{ij}} \leq \|R\inv-R\tp\|} \leq
        \|I-R\tp R\|\|R\inv\|$. Hence we need to construct bounds on
        $\|I-R\tp R\|$ by using the orthogonality of $Z + \delta Z$. 
        {\color{blue} \sffamily Why? The first inequality}
        \item Finally assembly these together, we have $\|Z-Q\| \leq h_1v +
        h_2 w$, if $Z$ is orthogonal at precision $v$ and $Q$ is its
        orthogonal QR factor computed by MGS at precision $w$.
    \end{itemize}    
\end{proof}

\begin{theorem}
    Theorem 4.2 Suppose that $A = ZDZ\tp$ is computed by any eigensolver
    in LAPACK or EISPACK in precision $v$ and $Q \in \mathbb{R}^{n
    \times n}$ is computed by using the MGS method to $Z$ in 
    precision $\omega(\omega \ll v)$. Then there exists a constant
    $\gamma_1 \equiv \gamma_1(n)$ such that 
    $$
        \|\operatorname{off}\left(Q^T A Q\right)\|_F \leq \gamma_1\|A\| v
    $$
\end{theorem}

\begin{proof}
    {Outline of proof}
    \ 

    \begin{itemize}
        \item Notice the important inequality: $\fnorm{\off(Q\tp AQ)}
        \leq \fnorm{Q\tp A Q - D}$\footnote{This is due to the $\off$
        operator remove all the diagonals, whereas $Q\tp A Q - D$ will
        not result all the diagonal entries to be zero. In fact, this is
        true for all diagonal matrices $D$, and the equality is attained
        for $D = \diag(\diag(Q\tp AQ))$.}.
        \item In order to relate this theorem with
        lemma~\ref{lemma.4-1}, we write 
        \begin{equation}\notag
            \fnorm{Q\tp A Q - D} \leq \fnorm{(Q-Z)\tp A (Q-Z)} 
            + 2\fnorm{(Q-Z)\tp AZ} + \fnorm{Z\tp A Z - D}
        \end{equation}
        \item We have already construct a bound on $\fnorm{Q-Z}$ by
        lemma~\ref{lemma.4-1}. Remain to construct bound on $\fnorm{Z\tp
        A Z - D}$.
        \item Notice the expression $D=(Z+\delta Z)\tp (A + E)(Z +
        \delta Z)$ from lemma~\ref{lemma.2-2}. Since we have the bound
        for $\norm{Z}, \norm{\delta Z}, \norm{E}$ and $\norm{Q-Z}$, we
        can construct the finalized bound.
    \end{itemize}
\end{proof}

The following two theorems gives the sufficient condition(s) on the
quadratic convergence of apply Jacobi algorithm to the preconditioned
matrix $T\iter{0} = Q\tp A Q$.

\begin{theorem}\label{thm.4-3}
    Theorem 4.3 Let $T^{(k)}$ be the matrix $T$ after $k$ Jacobi updates
    in Algorithm $4.1$ with $T^{(0)}=$ $Q^T A Q$. Suppose $\omega \ll v$
    and 
    $$
        \min_{\lambda_i\left(T^{(0)}\right) 
        \neq \lambda_j\left(T^{(0)}\right)}
        \left|\lambda_i\left(T^{(0)}\right)
        -\lambda_j\left(T^{(0)}\right)\right| \geq d>0 
    $$
    \bf{If $\gamma_1\|A\| v<d /(4 \sqrt{2})$} for some constant
    $\gamma_1=\gamma_1(n)$, then we have 
    $$
        \|\operatorname{off}\left(T^{(N)}\right)
        \|_F \leq \frac{\sqrt{34 / 9}}{d}
        \|\operatorname{off}\left(T^{(0)}\right)\|_F^2,
    $$
    where $N=n(n-1) / 2$.
\end{theorem}

\begin{proof}
    [Outline of proof]
    We know that from previous theorem that, there exist a $\gamma_1$,
    such that 
    \begin{equation}\notag
        \fnorm{\off(Q\tp A Q)} 
        = \fnorm{\off(T\iter{0})}\leq \gamma_1 \norm{A}v.
    \end{equation}
    If we have $\gamma_1 \norm{A}v \leq d/(4\sqrt{2})$, then we have 
    \begin{equation}\notag
        \fnorm{\off(T\iter{0})} \leq d/(4\sqrt{2})
    \end{equation}
    apply the Lemma~\ref{lemma.3-3} with $k = 0$, we have 
    \begin{itemize}
        \item $\Egap{T\iter{0}} \geq d$ (by assumption)
        \item $\fnorm{\off(T\iter{0})} \leq d/4\sqrt{2}$ by assumption
        and previous theorem.
    \end{itemize}
    Hence we have the quadratic convergence.
\end{proof}

\begin{theorem}
    Theorem 4.4 Let $T^{(k)}$ be the matrix $T$ after $k$ Jacobi updates in Algorithm $4.1$ with $T^{(0)}=$ $Q^T A Q$. Suppose $\omega \ll v$ and
    $$
    \min _{\lambda_i(A) \neq \lambda_j(A)}\left|\lambda_i(A)-\lambda_j(A)\right| \geq d>0 .
    $$
    If $2 \gamma_2\|A\| \omega \leq \rho_1 d$ for some constant $\gamma_2=\gamma_2(n)$ and $0<\rho_1<1$, and $\gamma_1\|A\| v<\left(1-\rho_1\right) d /(4 \sqrt{2})$ for some constant $\gamma_1=\gamma_1(n)$, then we have
    $$
    \left\|\operatorname{off}\left(T^{(N)}\right)\right\|_F \leq \frac{\sqrt{34 / 9}}{\left(1-\rho_1\right) d}\left\|\operatorname{off}\left(T^{(0)}\right)\right\|_F^2
    $$
\end{theorem}

\begin{proof}
    Let $d_A \leq \Egap{A}$, and 
    suppose we have 
    \begin{equation}\notag
        2\gamma_1 \norm{A} w \leq \rho_1 d_A
    \end{equation}
    for some $0\leq \rho_1 \leq 1$, then we have 
    \begin{equation}\label{eq.11}
        (1-\rho_1)d_{A} \leq d_{T\iter{0}}.
    \end{equation}
    The sufficient condition for the algorithm to converge that related
    to $d_{T\iter{0}}$ is the following:
    \begin{equation}\label{eq.12}\tag{AIM}
        \gamma_1 \norm{A} v \leq \frac{d_{T\iter{0}}}{4\sqrt{2}}.
    \end{equation}
    From equation~\eqref{eq.11}, we have 
    \begin{equation}\label{eq.13}
        \frac{(1-\rho_1)d_A}{4\sqrt{2}} 
        \leq \frac{d_{T\iter{0}}}{4\sqrt{2}}.
    \end{equation}
    In order to satisfies \eqref{eq.12}, we need 
    \begin{equation}\notag
        \gamma_1\norm{A}v \leq \frac{(1-\rho_1)d_A}{4\sqrt{2}}.
    \end{equation}
    Hence if we precondition the matrix $A$ by $Q\tp A Q$, then by
    Thm~\ref{thm.4-3}, 
    \begin{equation}\notag
        \fnorm{\off(Q\tp A Q)} = \fnorm{\off(T\iter{0})} 
        \leq \gamma_1\norm{A}v \leq \frac{(1-\rho_1)d_A}{4\sqrt{2}}
        \leq \frac{d_{T\iter{0}}}{4\sqrt{2}},
    \end{equation}
    which ensures the quadratic convergence by relating to the eigengap
    of $A$ rather than $Q\tp A Q$.
\end{proof}


%%%%%%%%%%%%%%%%%%%%%%%%%%%%%%%%%%%%%%%%%%%%%%%%%%%%%%%%%%%%%%%%%%%%%%%%
\section{Supplements}

In the proof of this lemma, there is a statement:
\begin{claim}{}{}
    Suppose $Z$ and $Q$ are the matrices from Lemma~4.1, then 
    \begin{align}
        & \lambda_{\max}(Z\tp Z) \leq 
        1 + (2\norm{\delta Z}\norm{Z} + \norm{\delta Z}^2),
        \label{eq.1.1}\\
        & \lambda_{\min}(Z\tp Z) \geq 
        1 - (2\norm{\delta Z}\norm{Z} + \norm{\delta Z}^2).
        \label{eq.1.2}
    \end{align}
\end{claim}
We will present a proof of this:
\begin{proof}
    We need the following result. From Lemma~\ref{lemma.1.1}, we have $Z
    + \delta Z$ is orthogonal. Moreover, we have 
    \begin{equation}\label{eq.1.3}
        \norm{Z}\leq\norm{Z+\delta Z}+{\delta Z} = 1+\norm{\delta Z}.
    \end{equation}

    \emph{For Eq.~\eqref{eq.1.1}.} From \eqref{eq.1.3}, we have
    $\norm{Z} - \norm{\delta Z} \leq 1$. Then taking the square of both
    sides, we have 
    \begin{equation}\notag
        \norm{Z}^2 - 2\norm{Z}\norm{\delta Z} + \norm{\delta Z}^2 \leq 1
    \end{equation}
    Moving around the equation, we have 
    \begin{equation}\notag
        \lambda_{\max}(Z\tp Z) = \norm{Z}^2 \leq 
        1 + 2\norm{Z}\norm{\delta Z} - \norm{\delta Z}^2 \leq 
        1 + 2\norm{Z}\norm{\delta Z} + \norm{\delta Z}^2.
    \end{equation}

    \emph{For Eq.~\eqref{eq.1.2}.}  Noticing the following lemma 
    \begin{lemma}
        If $A$ and $A + E$ are $n\times n$ real symmetric matrices, then 
        \begin{equation}\notag
            \abs{\lambda_k(A + E) - \lambda_k(A)} \leq \norm{E}
        \end{equation}
        for $k = 1,2,\dots,n$.
    \end{lemma}

    Let us consider $(Z + \delta Z)\tp(Z + \delta Z) = Z\tp Z + Z\tp
    \delta Z + \delta Z\tp Z + \delta Z \tp \delta Z$. If we let $A+E
    =(Z + \delta Z)\tp(Z + \delta Z)$ and $A=Z\tp Z$, then apparently,
    this $A$ and $A + E$ satisfies the condition of above lemma, then 
    \begin{equation}\notag
        \begin{aligned}
            1 = \lambda_{\min}((Z + \delta Z)\tp(Z + \delta Z)) 
            & \leq \lambda_{\min}(Z\tp Z) + \norm{Z\tp
            \delta Z + \delta Z\tp Z + \delta Z \tp \delta Z} \\
            & \leq \lambda_{\min}(Z\tp Z)  + 
            (2\norm{Z}\norm{\delta Z} + \norm{\delta Z}^2) \\
            \implies \lambda_{\min}(Z\tp Z) & \geq 
            1 - (2\norm{Z}\norm{\delta Z} + \norm{\delta Z}^2)
        \end{aligned}
    \end{equation}
    which proves the claim.
\end{proof}

%%%%%%%%%%%%%%%%%%%%%%%%%%% Bibliographies %%%%%%%%%%%%%%%%%%%%%%%%%%%

\bibliographystyle{nj-plain}
\bibliography{bib.bib}


\end{document}